\documentclass[11pt]{amsbook}

\usepackage[pdftex]{graphicx}
\usepackage{../HBSuerDemir}	% ------------------------

%\usepackage[update,prepend]{epstopdf}

\usepackage{epstopdf}
\epstopdfsetup{outdir=./images}
\epstopdfsetup{update} % only regenerate pdf files when eps file is newer

\setcounter{tocdepth}{3}

\usepackage{fancyhdr} % Header/Footer
\pagestyle{fancy}
%\fancyhead{}
%\fancyhead[CO,CE]{óDraftó}
\fancyfoot{}
%\fancyfoot[CO,CE]{Draft}
%\fancyfoot[RO,LE] {\thepage}
\fancyfoot[L]{\footnotesize 
Freshman Calculus by Suer \& Demir  \textbf{DRAFT} \\
\LaTeX  by Haluk Bingol 
\href{http://www.cmpe.boun.edu.tr/~bingol}
{http://www.cmpe.boun.edu.tr/~bingol} 
%\large 
%\footnotesize 
\today}
\fancyfoot[R]{{\thepage} of \pageref{LastPage}}


\begin{document}



% -------------------------------------------------------------------
\chapter{XXX}
\label{chap:FuncLimCont}




% -------------------------------------------------------------------
\section{XXX}
\label{sec:Numbers}


%: b2p1/023 ++++++++++++++++++++++++++++++++++++++
\hPage{b2p1/023}
% ++++++++++++++++++++++++++++++++++++++
\begin{exmp}
Test the series $\sum_2 \frac{1}{n \ln^{p} n} $ for convergence.\\
	\begin{hSolution}
		The series is of positive terms. 
		The function $f(x) = \frac{1}{x \ln^{p} x}$ fulfils 
		the conditions of integral test on $[2, \infty)$:
		\[
			\int_{2}^{\infty} \frac{\hDif x}{x\ ln^{p} x}
			= \int_{2}^{\infty} \frac{\hDif \ln x}{\ln^{p}_{n} x}
			= \frac{1}{1 - p} \ln^{\frac{1}{x} - p} \lvert^{\infty}_{2}
		\]
		converging when $p>1$, 
		diverging when $p \le1$. 
		Hence the given series is convergent only when $p > 1$.
	\end{hSolution}
\end{exmp}




% =======================================
\subsubsection{Comparison with other series}

Below we give two tests by comparison of a given series of 
positive terms with other such series.
\begin{thm}[Test by inequality]
	\label{thm:TestByInequality}
	Let $\sum a_n$ be a given series of positive terms, and 
	let $\sum c_n$, $\sum d_n$ be two such series 
	which are convergent and divergent, respectively. 
	Then $\sum a_n$ is convergent or divergent according as
	\[
		a_n \le c_n \quad \text{or}  \quad a_n\ge d_n
	\]
	for all $n > N$ for some $N$.

	\begin{proof}
		$a_n \le c_n \Rightarrow \sum_N^n a_n \leq \sum_N^n c_n$ . 
		By hypothesis $\sum c_n$ having a limit as $n \rightarrow \infty $, 
		the sequence $(\sum_N^n a_n)$ is bounded above by this limit. 
		Since it is increasing, 
		has a limit, and $\sum a_n$ is convergent.\\
		Divergence case can be proved similiarly.
	\end{proof}
\end{thm}

A generalization of the above \refthm{thm:TestByInequality}:
The Theorem holds true when the inequalities are replaced by
\[
	a_n \leq p  c_n \text{ or } a_n \geq q d_n
\]
where $p$ and $q$ are positive numbers.

%: b2p1/029 ++++++++++++++++++++++++++++++++++++++
\hPage{b2p1/029}
% ++++++++++++++++++++++++++++++++++++++
\begin{thm}
	XXXXXXX
	\begin{proof}
		It will suffice to prove that $s_{2n}$ and $s_{2n+1}$ have the same limit
		\begin{align*}
			s_{2n} 
				&= (a_{0} - a_{1})
				+ ... +(a_{2n} - a_{2n})
				\geq 0 
				\tag{from 1}\label{myeq}\\
			s_{2n} 
				&= a_{0} 
				- (a_{1} - a_{2})
				- \dotsb
				- (a_{2n-2} - a_{2n-1})
				- a_{2n} 
				\leq a_{0} 
				\tag{from 1}\label{mye1q}\\
			\Longrightarrow 
				&0 \leq s_{2n} \leq a_{0}.
		\end{align*}
%		\[
%			s_{2n} 
%				= (a_{0} - a_{1})
%				+ ... +(a_{2n} - a_{2n})
%				\geq 0 
%				\tag{from 1}\label{myeq}
%		\] 
%		\[
%			s_{2n} 
%				= a_{0} 
%				- (a_{1} - a_{2})
%				- \dotsb
%				- (a_{2n-2} - a_{2n-1})
%				- a_{2n} 
%				\leq a_{0} 
%				\tag{from 1}\label{mye1q}
%		\]
		
Hence $(s_{2n})$ is bounded, and being monotone increasing it converges to a limit s. 
Now,
		\[
			s_{2n+1} 
				= s_{2n}+a_{2n+1} + s + 0= s.
		\]
	\end{proof}
\end{thm}
\begin{cor}
	In a convergent alternating series
	\[
		s 
			= a_{0} - a_{1} + a_{2} - \dotsb + (-1) a_{n} + R_{n+1}
	\]
	with given hypothesis, the inequality
	\[
		\hAbs{R_{n+1}} < a_{n+1}
	\]
	holds, that is the error made in taking $s_{n}$ for s is less than $a_{n+1}$.
	\begin{proof}
		\begin{align*}
			R_{n+1} 
				&= (-1)^{n+1} (a_{n+1} - a_{n+2} + \dotsb)\\
			\Longrightarrow \hAbs{R_{n+1}}
				&= \hAbs{ a_{n+1} - a_{n+2} + \dotsb}\\
				&= a_{n+1} - (a_{n+2} - a_{n+3}) - \dotsb < a_{n+1}.
		\end{align*}
	\end{proof}
\end{cor}
\begin{exmp}
	Given the alternating harmonic series	
	\[
		1 - \frac{1}{2} + \frac{1}{3} 
		- \dotsb 
		+ (-1)^{n+1} \frac{1}{n} 
		+ \dotsb
	\]
	\begin{hEnumerateAlpha}
		\item
		show its convergence
		
	\end{hEnumerateAlpha}
\end{exmp}



%: b2p1/031 ++++++++++++++++++++++++++++++++++++++
\hPage{b2p1/031}
% ++++++++++++++++++++++++++++++++++++++
A series $\sum a_{n}$ such that $\sum|a_{n}|$ is convergent is called an
\hDefinedC{absolutely convergent series}, and
the above theorem states 
that an absolutely convergent series is convergent.

As the alternating harmonic series shows, a series may be
convergent without being absolutely convergent. Such series are
called \hDefinedC{simply convergent}~\footnote{
	In many textbooks 
	\hDefinedC{conditional convergent} or 
	\hDefinedC{semi-convergent} 
	terminologies are used instead of simply convergent.
} 
series:
\begin{align*}
    \sum|a_{n}| \text{ (conv.)} 
        &\Longrightarrow 
        \sum a_{n} \text{ (conv)} \dots \text{ abs. conv. of  } \sum a_{n}\\
    \sum|a_{n}| \text{ (div.)} 
        &\Longrightarrow 
        \left\{
            \begin{array}{ll}
             \sum a_{n} \text{ (conv)} \dots \text{ simply. conv. of  } \sum a_{n}\\
             \text{or}\\
             \sum a_{n} \text{ (div)} 
    \end{array}
    \right.    
\end{align*}
There is an essential difference between the absolutely convergent series and simply convergent ones. 
The absolutely convergent series have the following two properties among others:

\begin{enumerate}

    \item 
    The terms can be rearranged in any order 
    (rearrangement does not alter the sum).
    
    \item 
    Finitely or infinitely many terms may be replaced by their sum.
\end{enumerate}

These properties may not be shared by simply convergent series, 
that is, a rearrengement of terms in a simply convergent
series may give a different sum as illustrated by the following example:

Consider the simply alternating harmonic series
\[
    S
    = 
    1 - \frac{1}{2} + \frac{1}{3} - \frac{1}{4} + \dots + (-1)^{n+1} \frac{1}{n} + \dots
\]
Let us rearrange the terms to have the series 
\begin{align*}
	S'
	=
	&\left( 1 - \frac{1}{2} - \frac{1}{4} \right)
	+ \left( \frac{1}{3} - \frac{1}{6} - \frac{1}{8} \right) 
	+ \left( \frac{1}{5} - \frac{1}{10} - \frac{1}{12} \right)\\
	&+ \dots
	+ \left(\frac{1}{2n+1} - \frac{1}{4n+2} -\frac{1}{4n+4} \right) 
	+ \dots 
\end{align*}



% ========================================
\end{document}  

%==== templates ====

%==== environments ====

%\begin{figure}[htb]
%	\centering
%	\includegraphics[width=0.9\textwidth]{images/SD-1-1p15A}
%	\caption{Classification of complex numbers}
%	\label{fig:classificationOfComplexNumbersA}
%\end{figure}

%\begin{center}
%\begin{tabular}{cc}
%\end{tabular}
%\end{center}

%\begin{exmp}
%\begin{hSolution}
%\end{hSolution}
%\end{exmp}

%\begin{hEnumerateAlpha}
%\end{hEnumerateAlpha}

%\begin{hEnumerateRoman}
%\end{hEnumerateRoman}

%$
%\begin{bmatrix}
%\end{bmatrix}
%$

%\frac{aaaa}{bbb}
%\frac{a_{n}}{b_{n}}
%\left( aaaa \right)
%\Longrightarrow

% ++++++++++++++++++++++++++++++++++++++
\hPage{b1p1/xxx}
% ++++++++++++++++++++++++++++++++++++++

%\begin{multicols}{2}
%	bb
%\columnbreak
%	aa
%\end{multicols}
