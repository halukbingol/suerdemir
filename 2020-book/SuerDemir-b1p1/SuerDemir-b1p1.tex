\documentclass[11pt]{amsbook}

\usepackage[pdftex]{graphicx}
\usepackage{../HBSuerDemir}	% ------------------------

%\usepackage[update,prepend]{epstopdf}

\usepackage{epstopdf}
\epstopdfsetup{outdir=./images}
\epstopdfsetup{update} % only regenerate pdf files when eps file is newer

\setcounter{tocdepth}{3}

\usepackage{fancyhdr} % Header/Footer
\pagestyle{fancy}
%\fancyhead{}
%\fancyhead[CO,CE]{óDraftó}
\fancyfoot{}
%\fancyfoot[CO,CE]{Draft}
%\fancyfoot[RO,LE] {\thepage}
\fancyfoot[L]{\footnotesize 
Freshman Calculus by Suer \& Demir  \textbf{DRAFT} \\
\LaTeX  by Haluk Bingol 
\href{http://www.cmpe.boun.edu.tr/~bingol}
{http://www.cmpe.boun.edu.tr/~bingol} 
%\large 
%\footnotesize 
\today}
\fancyfoot[R]{{\thepage} of \pageref{LastPage}}


\begin{document}



% -------------------------------------------------------------------
\chapter{Function, Limit, Continuity}
\label{chap:FuncLimCont}




% -------------------------------------------------------------------
\section{Numbers}
\label{sec:Numbers}
%:b1p1/001 ++++++++++++++++++++++++++++++++++++++
\hPage{b1p1/001}
% ++++++++++++++++++++++++++++++++++++++




% -------------------------------------------------------------------
\subsection{Integers}
\label{subsec:Integers}

Following historical development, the earliest numbers were the 
\hDefined{counting numbers} 
$1, 2, 3, \dotsc, n, \dotsc$. 
Introducing the number zero, one obtain the numbers 
$0\ 1, 2, \dotsc, n, \dotsc$ 
called the 
\hDefined{natural numbers}. 
The natural numbers, except 0, that is, the counting numbers are all positive and are referred to as 
\hDefined{positive integers}. 
Assigning "-" sign to these numbers one gets the 
\hDefined{negatives integers}, namely, 
$-1, -2, -3, \dotsc$
A positive integer, a negative integer or zero is called an 
\hDefined{integer}.




% -------------------------------------------------------------------
\subsection{Rational Numbers}
\label{subsec:RationalNumbers}

Any number in the form of a ratio $p/q$ of two integers 
($p \ne 0$) is called a 
\hDefined{rational number} or a 
\hDefined{fraction}.
Any integer is a rational number ($p=p/1$). 
Thus 3/4, 17/5, -11/7, 6, -9 are rational numbers.

The decimal expansion of any rational number $p/q$ obtained by ordinary division is either finite or else infinite. It is know from Arithmetic that an infinite expansion of a rational number contains a repeating block as given in the following examples:
\begin{align*}
	0,19771977...1977...  \quad &(= 0,\overline{1977}) \\
	-5,112323...23.. \quad &(= -5,11\overline{23})
\end{align*}
A finite expansion can be considered as 
an infinite expansion with ``0'' as repeating block:
\[
	12,75 \quad (= 1275,\overline{0}) 
\]

%: b1p1/002 ++++++++++++++++++++++++++++++++++++++
\hPage{b1p1/002}
% ++++++++++++++++++++++++++++++++++++++
\begin{exmp}
	Find the (repeating) decimal expansion of 
	the rational number $152/55$.
	
	Dividing $152$ by $55$ one gets~\footnote{@HB needs correction}
	\[
		\begin{array}{lr}
		152|\frac{55}{2,76363\cdots 63\cdots =2,76\overline{63}} \\
		\frac{|110\quad}{420} \\
		\frac{|385\quad}{150} \\
		\end{array}
	\]
\end{exmp}

Conversely, any decimal expansion with repeating block (\hDefined{cyclic expansion}) represents a rational number.

\begin{exmp}
	Express the repeating decimal expansion 
	$3,71\overline{05}$ as a ratio of two integers.
	\begin{hSolution}
		Set $r = 3,71\overline{05}$.
		Multiply each side by $1000$ to bring ``,'' 
		just after the repeating block, 
		and also multiply each side by $100$ to bring ``,'' 
		just before the repeating block:
		\begin{center}
		\begin{tabular}{rlr}
			$10000 r$	&= 	&$37105,\overline{05}$ \\
			$100 r$ 	&= 	&$371,\overline{05}$ \\
%			\hline
%			$9900 r$	&= 	&$36734,00$
		\end{tabular}
		\end{center}
		
		Subtraction gives
		\begin{align*}
			9900 r &= 36734 \\
			r &= \frac{36734}{9900}
		\end{align*}
	\end{hSolution}
\end{exmp}

\begin{hProperty} 
If 
$r_{1} (=p_{1}/q_{1})$, 
$r_{2} (=p_{2}/q_{2})$ 
are two rational numbers, then the numbers 
%: b1p1/003 ++++++++++++++++++++++++++++++++++++++
\hPage{b1p1/003}
% ++++++++++++++++++++++++++++++++++++++
\noindent
\begin{hbColi}{2}
	
	\item 
	$r_{1}+r_{2}  
	\hPairingParan{\dfrac{p_{1}}{q_{1}} 
		+ \dfrac{p_{2}}{q_{2}}
	= \dfrac{p_{1} q_{2} + p_{2} q_{1}}
		{q_{1} q_{2}}}$,

	\item 
	$r_{1}-r_{2} 
	\hPairingParan{\dfrac{p_{1}}{q_{1}} 
		- \dfrac{p_{2}}{q_{2}}
	= \dfrac{p_{1} q_{2} - p_{2} q_{1}}
		{q_{1} q_{2}}}$,

	\item 
	$r_{1} \cdot r_{2}
	\hPairingParan{\dfrac{p_{1}}{q_{1}} 
		\cdot \dfrac{p_{2}}{q_{2}}
	= \dfrac{p_{1} p_{2}}
	{q_{1} q_{2}}}$,

	\item 
	$r_{1}:r_{2}
	\hPairingParan{\dfrac{p_{1}}{q_{1}} 
		: \dfrac{p_{2}}{q_{2}}
	= \dfrac{p_{1} q_{2}}
		{q_{1} p_{2}}}$
\end{hbColi} 
are all rational.
\end{hProperty}

\begin{cor}
	Between any two distinct rational numbers 
	there exists at least one rational number, 
	hence infinitely many.

	\begin{proof}
		Let the given rational numbers be $r_{1}$ and $r_{2}$ :
		$r_{1} + r_{2}$ rational 
		$\Longrightarrow$ 
		$\dfrac{1}{2}(r_{1} + r_{2})$ is rational.
		(why this arithmetic mean is between $r_{1}$ and $r_{2}$?) 
		
		This process can be continued indefinitely.
	\end{proof}
\end{cor}





% -------------------------------------------------------------------
\subsection{Irrational numbers}
\label{subsec:IrrationalNumbers}

A number which is not rational is called an 
\hDefined{irrational number}. 
Since any cyclic decimal expansion is a rational number, 
then non cyclic ones represent irrational numbers:
\begin{align*}
	0,81881888188881\cdots 
	&\text{ (Number of 8's increases by 1 in each step)} \\
	4,303003000300003\cdots 
	&
\end{align*}

The existence of irrational numbers may also be shown 
by the following theorem:
%: b1p1/004 ++++++++++++++++++++++++++++++++++++++
\hPage{b1p1/004}
% ++++++++++++++++++++++++++++++++++++++
\begin{thm}
	If $n$ is a positive prime number, 
	then $\sqrt{n}$ is irrational.

	\begin{proof}
		Suppose $\sqrt{n} = p / q$ 
		where the integers $p$, $q$ have 
		no common factor (divisor) other than 1. 
		Any fraction can be reduced into this form by simplification.
		\[
			\sqrt{n} = p / q  
			\Longrightarrow 
			q^{2} n = p^{2}
		\]
		Since $n \mid q^{2} n$ $(n \text{ divides } q^{2}n)$, 
		then 
		$n \mid p p$ implying $n \mid p$. 
		Therefore for some integer $k$ 
		we have $p = k n$.
		\[
			q^{2} n = k^{2} n^{2} 
			\Longrightarrow 
			k^{2} n = q^{2} 
			\Longrightarrow 
			n \mid q.
		\]
		The results $n \mid p$, $n \mid q$ show that 
		$p$, $q$ have a common factor $n ( > 1 )$, 
		contradicting the assumption that 
		$p$, $q$ had no common factor.
	\end{proof}
\end{thm}


Some irrational numbers of this form are 
\[
	\sqrt{2}, 
	\sqrt{3}, 
	\sqrt{5}, 
	\sqrt{7}
	\quad 
	\text{ (Why $\sqrt{4}$ is not irrational?)}
\]

\begin{hProperty}
	Let $r$ be a rational and $\alpha$ be an irrational number.
	Then 
	\begin{hbColi}{4}
		\item $r + \alpha$
		\item $r - \alpha$
		\item $r \alpha$
		\item $r / \alpha$
	\end{hbColi}
	are all irrational.

	\begin{proof}[Proof of (i)]
		Suppose that $r + \alpha$ is equal to a rational number $s$.
		Then, 
		$r + \alpha = s$ 
		$\Longrightarrow$ 
		$\alpha = s - r$
		$\Longrightarrow$ 
		$\alpha$ 
		is a rational number, 
		since $s - r$ is rational. 
		This contradicts the hypothesis.
		Hence $r + \alpha$ is irrational.
	\end{proof}
	The proofs of other cases can be done similarly. \\
\end{hProperty}


\begin{rem}
	The sum, difference, product and the ratio of two irrational numbers may not be an irrational number: \\
{\color{blue} % ~~~~
		\begin{multicols}{2}
		$(3 + \sqrt{2}) + (5 - \sqrt{2}) = 8$,\\
		$\hPairingParan{\dfrac{2}{3} + \sqrt{5}}
		\hPairingParan{\dfrac{2}{3} - \sqrt{5}} 
		= -\dfrac{41}{9}$,\\
	\columnbreak
		$(3 + \sqrt{2}) - (5 + \sqrt{2}) = -2$,\\
		$\sqrt{18} / \sqrt{2} = 3$.
	\end{multicols}
} % color ~~~~

	\begin{tabular}{ll}
		$(3 + \sqrt{2}) + (5 - \sqrt{2}) = 8$, 
		& $(3 + \sqrt{2}) - (5 + \sqrt{2}) = -2$, \\
		$\hPairingParan{\dfrac{2}{3} + \sqrt{5}}
		\hPairingParan{\dfrac{2}{3} - \sqrt{5}} 
		= -\dfrac{41}{9}$, \quad \quad
		& $\sqrt{18} / \sqrt{2} = 3$.
	\end{tabular}
\end{rem}

\begin{cor}
	Between any two distinct rational numbers, 
	there exists at least one irrational number, 
	and hence infinitely many.

	\begin{proof}		
		Let the given rational numbers be $r_{1}$ and $r_{2}$ 
		($r_{1} < r_{2}$).
		$\sqrt{2}$ being irrational, 
		for a sufficiently large positive integer $m$, 
		the irrational number $\sqrt{2} / m$ is less than 
		the difference $r_2 - r_{1}$. 
		Then $r_{1} + (\sqrt{2} / m)$ is irrational and 
		lies between $r_{1}$ and $r_{2}$.
		
		For all integers $n > m$ the irrational numbers 
		$r_{1} + (\sqrt{2}/n)$ lie between $r_{1}$ and $r_{2}$. 
	\end{proof}
\end{cor}




% -------------------------------------------------------------------
\subsection{Real numbers}
\label{subsec:RealNumbers}

A rational or an irrational number is called a 
\hDefined{real number}. 

The four arithmetic operations (rational operations) 
for any two real numbers will always yield  real numbers 
(excluding the case $a / b$ where $b = 0$).

The above definition provides a classification of real numbers as rational and irrational. 
Real numbers can also be classified as algebraic and non-algebraic (transcendental) numbers: 
The roots of polynomial equation 
\[
	a_{0} x^{n} + \dotsb + a_{n-1} x + a_{n} = 0
\]
with rational coefficients are called 
\hDefined{algebraic numbers}, 
and non algebraic real numbers are called 
\hDefined{transcendental numbers}.

According to this definition all rational numbers are algebraic 
$(x - p / q = 0)$. 
%According to this definition all rational numbers are algebraic 
%$\hPairingParan{x - \dfrac{p}{q} = 0}$. 
Some irrational numbers which are algebraic are 
$\sqrt{2}$, 
$5 - \sqrt{3}$; for 
$x = \sqrt{2} 
\Longrightarrow 
x^{2} - 2 = 0$ and 
%: b1p1/006 ++++++++++++++++++++++++++++++++++++++
\hPage{b1p1/006}
% ++++++++++++++++++++++++++++++++++++++
$x = 5 - \sqrt{3}$ 
$\Longrightarrow$ 
$(x-5)^{2} = 3$ 
$\Longrightarrow$ 
$x^{2} - 10 x + 22 = 0$.
Some irrational numbers which are transcendental 
are the well known numbers $\pi$ and 
the base $e$ of natural logarithm.




% -------------------------------------------------------------------
\subsubsection{Real number axis}
\label{subsubsec:RalNumberAxis}

A line (straight line) on 
which real numbers are represented in some manner is called a 
\hDefined{real number axis} 
or shortly a 
\hDefined{number axis}. 
In general a representation is done by choosing on the axis 
a fixed point 0 as origin corresponding to zero, 
a positive sense, and 
a unit length to locate first, 
integers in succession as seen in Fig~\ref{fig:numberAxis}.

\begin{figure}
\center
	\begin{picture}(60,15)(-25,-10)
		\setlength{\unitlength}{2pt}
	% \graphpaper(0,0)(80,70)
		\put(25,-2){\makebox(0,0)[tl]{$X$}}
		\put(-25,0){\vector(1,0){50}} % x-axis
		\multiput(-20,-1)(10,0){5}{\line(0,1){2}} % x-axis
	%
		\put(-20,-2){\makebox(0,0)[tc]{$-2$}}
		\put(-10,-2){\makebox(0,0)[tc]{$-1$}}
		\put(0,-2){\makebox(0,0)[tc]{$0$}}
		\put(10,-2){\makebox(0,0)[tc]{$1$}}
		\put(20,-2){\makebox(0,0)[tc]{$2$}}
	\end{picture}
	\caption{Number axis}
	\label{fig:numberAxis}
\end{figure}

By the use of Thales Theorem, 
a rational number $p/q$ can be constructed on the number axis. 
To find the point on the number axis 
corresponding to the number $p/q$, 
a ray $OT$ (non parallel to $0X$) is drawn 
on which line segments $[OP]$, $[OQ]$ of lengths $p$, $q$ units are taken (Fig~\ref{fig:constructionRationalNumber}). 
Then $Q$ is joined to the point represented by 1. 
The line passing through $P$ and parallel to $[Q1]$ intersects the number axis at the required point.

When $p < q < 0$, the point $Q$ is joined to 
the point representing -1 instead of 1.

\begin{figure}
	\center
	\begin{picture}(60,15)(-25,-10)
	\end{picture}
	\caption{Construction of a rational number}
	\label{fig:constructionRationalNumber}
\end{figure}

%: b1p1/007 ++++++++++++++++++++++++++++++++++++++
\hPage{b1p1/007}
% ++++++++++++++++++++++++++++++++++++++
%: b1p1/008 ++++++++++++++++++++++++++++++++++++++
\hPage{b1p1/008}
% ++++++++++++++++++++++++++++++++++++++
%: b1p1/009 ++++++++++++++++++++++++++++++++++++++
\hPage{b1p1/009}
% ++++++++++++++++++++++++++++++++++++++
The positive square root of $a ( > 0)$ is denoted by $\sqrt{a}$ and 
the negative one by $- \sqrt{a}$. 
Thus, 
\[
	\sqrt{4} = 2, \quad 
	-\sqrt{4} = -2, \quad 
	\sqrt{(-3)^{2}} = \sqrt{9} = 3.
\]

The number $0$ which is neither positive nor negative 
has only one square root, namely $0$, 
as a double root of $x^{2} = 0$.




% -------------------------------------------------------------------
\subsubsection{Absolute Value}
\label{subsubsec:AbsoluteValue}

The \hDefined{absolute value} of a real number $a$ is 
a non-negative real number, denoted by $\hAbs{a}$ and defined by
\[
	\hAbs{a} = \sqrt{a^2} \quad (\geq 0)
\]
or equivalently, by
\[ 
	\hAbs{a} 
	= 
	\begin{cases}
		a, 	&a > 0, \\
		0,	&a = 0, \\
		-a,	&a < 0.
	\end{cases}
\]

The equivalency of two definitions can be seen 
by considering three cases 
$a > 0, \ a=0, \ a < 0$ separately.

\begin{tabular}{ll}
$\hAbs{5} = \sqrt{5^2} = 5$, 
&$\hAbs{-3} = \sqrt{(-3)^2} = \sqrt{9} = 3$ \\
$\hAbs{-2} = -(-2) = 2$, \quad \quad
& $\hAbs{2} = 2$
\end{tabular}

As an immediate corollary we have

\begin{cor} 
	\begin{tabular}{ll}
	1. $\hAbs{a}^{2} = a^{2}$ \quad \quad
	2. $-\hAbs{a} \leq a \leq |a|$.
	\end{tabular} 
\end{cor}
Some other properties are stated in the next theorem.
\begin{thm}
	If $a$, $b$ are real numbers, then \\
	%: b1p1/010 ++++++++++++++++++++++++++++++++++++++
	\hPage{b1p1/010}
	% ++++++++++++++++++++++++++++++++++++++
	\begin{tabular}{ll}
		1. $\hAbs{ab} = \hAbs{a} \hAbs{b}$ 
		&2. $\hAbs{\frac{a}{b}} = \frac{\hAbs{a}}{\hAbs{b}}$\\ 
		3. $\hAbs{a + b} \leq \hAbs{a} + \hAbs{b}$ \quad \quad
	\end{tabular}
{\color{blue} % ~~~~
		<<<< HERE
} % color ~~~~
	\begin{hbColi}{2}
		\item $\hAbs{ab} = \hAbs{a} \hAbs{b}$ 
		\item $\hAbs{\frac{a}{b}} = \frac{\hAbs{a}}{\hAbs{b}}$
		\item $\hAbs{a + b} \leq \hAbs{a} + \hAbs{b}$ \quad \quad
	\end{hbColi}

	\begin{proof}
	\begin{enumerate}
	
		\item [1.]
		$\hAbs{ab} 
		= \sqrt{(ab)^{2}}  
		= \sqrt{a^{2}}\sqrt{b^{2}} 
		= \hAbs{a} \hAbs{b}$
		
		\item [2.]
		Proved similarly.
		
		\item [3.]
		\begin{align*}
			\hAbs{a + b}^{2} 
			&= (a+b)^{2} \\
			&= a^{2} + 2ab + b^{2} \\
			&= \hAbs{a}^{2} + 2ab + \hAbs{b}^{2} \\
			&\leq \hAbs{a}^{2} + 2 \hAbs{a} \hAbs{b} + \hAbs{b}^{2} \\
			&= (\hAbs{a} + \hAbs{b})^{2} \\
			\hAbs{a + b}^{2} 
			&\leq (\hAbs{a} + \hAbs{b})^{2} 
		\end{align*}
		where 
		$\hAbs{a + b}$, 
		$\hAbs{a} + \hAbs{b}$ 
		being non-negative, taking positive square roots of each side,
		\[
			\hAbs{a + b} \leq \hAbs{a} + \hAbs{b}
		\]
		follows.
	\end{enumerate}
	\end{proof}
\end{thm}

Changing $b$ to $-b$ in the last inequality 
the latter is seen to include the inequality:
\[
	\hAbs{a + b} \leq \hAbs{a} + \hAbs{b}.
\]




% -------------------------------------------------------------------
\subsubsection{Distance}
\label{subsubsec:Distance}

The \hDefined{distance} between two points $A$ and $B$ with coordinates $a$, $b$ on the number axis, denoted by
\[
	d(A, B)= d(a, b) = \hAbs{AB},
\]
is defined as the non negative real number $\hAbs{b - a}$.

%: b1p1/011 ++++++++++++++++++++++++++++++++++++++
\hPage{b1p1/011}
% ++++++++++++++++++++++++++++++++++++++

\begin{exmp} 
	\begin{tabular}{lll}
		& $d(3, 5) = \hAbs{5 - 3} = 2$
		& $d(3, 5) = \hAbs{3 - 5} = 2$ \\
		& $d(3, -5) = 3 + 5 = 8$ \quad \quad
		& $d(-2, 7) = \hAbs{7 + 2} = 9$
	\end{tabular}
\end{exmp}




% -------------------------------------------------------------------
\subsection{Complex numbers}
\label{subsec:ComplexNumbers}

The roots of the quadratic equation 
\[
	a x^{2} + bx + c = 0 \quad (a \ne 0)
\]
with real coefficients, are given by
\[
	x_{1,2} = \frac{-b \pm \sqrt{b^{2} - 4ac}}{2a}
\]

They are real if and only if  (iff) the discriminant
$\Delta = b^{2} - 4ac$ is non negative. 
Then for a real $k$, if $\Delta = -k^{2} < 0$ the roots become non real and have the form
\[
	x_{1,2} = \frac{-b \pm ki}{2a} = u + iv
\]
where $u$ and $v$ are real numbers and 
$i=\sqrt{-1}$, unit imaginary number, with $i^{2}$ = -1.

Hence in general case for any $\Delta$ the roots of a quadratic equation are numbers of the form
\[
	u + iv
\]
which is called a \hDefined{complex number}.

A complex number
\[
	z = a + i b
\]
%: b1p1/012 ++++++++++++++++++++++++++++++++++++++
\hPage{b1p1/012}
% ++++++++++++++++++++++++++++++++++++++
is real or imaginary according as $b = 0$ or $b \ne 0$. 
The real numbers $a$ and $b$ are called, respectively, the \hDefined{real part} and 
\hDefined{imaginary part} of $z$, written
\[
	a = \text{Re } z, \quad b = \text{Im } z.
\]

\begin{defn}[Equality]
	Two complex numbers are equal 
	iff 
	their real parts are equal and imaginary parts equal:
	\[
		a + ib = c + id  
		\iff
%		\hDefinitionIFF   
		a = c, b = d.
	\]
	Hence $a + ib = 0$ $\iff$  $a = 0, b = 0$.
\end{defn}

\begin{defn}[Conjugation]
	If $z = a + ib$, then the number $a - ib$ is called the 
	\hDefined{complex conjugate} or simply 
	\hDefined{conjugate} of z, 
	written $\overline{z} =a - ib$. 
\end{defn}

From $a +ib = a - ib \implies b = 0$, it follows that a complex number is real iff it is equal to conjugate:
\[
	z = \overline{z} 
	\iff 
	z \text{ is real}.
\]

\begin{defn}[Addition and subtraction]
	If $z_{1} = a_{1} + ib_{1}$, $z_{2} = a_{2} + ib_{2}$, 
	then their 
	\hDefined{sum} and 
	\hDefined{difference} 
	are defined as follows:
	\begin{center}
	\begin{tabular}{c}
		1. $z_{1} + z_{2} = a_{1} + a_{2} + i( b_{1} + b_{2} )$,\\
		2. $z_{1} - z_{2} = a_{1} - a_{2} + i( b_{1} - b_{2} )$.
	\end{tabular}
	\end{center}
\end{defn}

One concludes that 
\[
	\overline{ z_{1} + z_{2} } 
	= \overline{z_{1}} + \overline{z_{2}},
	\quad
	\overline{ z_{1} - z_{2} } 
	= \overline{z_{1}} - \overline{z_{2}}.
\]
In words: 
The conjugate of a sum (difference) is the sum (difference) of conjugates.

A complex number is multiplied by a real scalar $k$ 
by multiplying its real and imaginary parts by $k$:
%: b1p1/013 ++++++++++++++++++++++++++++++++++++++
\hPage{b1p1/013}
% ++++++++++++++++++++++++++++++++++++++
\[
	k(a + ib) = ka + ikb.
\].

\begin{exmp}
	Simplify
	\begin{hEnumerateAlpha}
		\item 
		$u = (2 - 3i) - 2(4 + 2i)$, 
		
		\item 
		$v = \overline{2(3 - 2i) + 3i}$.
	\end{hEnumerateAlpha}
	\begin{hSolution}
		\begin{hEnumerateAlpha}
			
			\item 
			$u 
			= 2 - 3i - 8 - 4i 
			= 2 - 8 - (3i + 4i) 
			= -6 - 7i$. 
			
			\item 
			$v 
			= \overline{6 - 4i + 3i} 
			= \overline{6 - i} 
			= 6 + i$.
		\end{hEnumerateAlpha}
	\end{hSolution}
\end{exmp}

\begin{defn}[Multiplication]
	The \hDefined{product} of two complex numbers 
	is obtained as follows:
	\begin{align*}
		(a + ib)(c + id) 
		&= ac + iad + ibc + i^{2}bd \\
		&= ac + i(ad + bc) - bd \quad (\text{Note that } i^{2} = -1)\\
		&= (ac - bd) + i(ad + bc)
	\end{align*}
\end{defn}

\begin{cor}
	$z = a + ib$ 
	$\Longrightarrow$ 
	$z \overline{z} = a^{2} + b^{2}$.
\end{cor}

\begin{exmp}
	Perform multiplications:
	\begin{hEnumerateAlpha}
		
		\item 
		$u = (2 - 3i)(5 + i)$,
		
		\item 
		$v = (2 - 3i)(2 + 3i)$.
	\end{hEnumerateAlpha}
	\begin{hSolution}
		\begin{hEnumerateAlpha}
			
			\item 
			$u = 10 + 2i - 15i - 3i^{2}
			= 10 - 13i + 3 
			= 13 - 13i$.
			
			\item 
			$v = (2 - 3i)(2 + 3i) 
			= 2^{2} + 3^{2} 
			= 4 + 9 = 13$.
		\end{hEnumerateAlpha}
	\end{hSolution}
\end{exmp}

\begin{defn}
	In view of above corollary, 
	\hDefined{division $u/v$} is carried out 
	by multiplying the numerator and denominator 
	by the conjugate $\overline{v}$ of the denominator:
	\[
		\dfrac{u}{v} 
		= \dfrac{u}{v} \dfrac{\overline{v}}{\overline{v}}
		= \dfrac{1}{v \overline{v}} u \overline{v}.
	\]
\end{defn}
%: b1p1/014a ++++++++++++++++++++++++++++++++++++++
\hPage{b1p1/014a}
% ++++++++++++++++++++++++++++++++++++++




% -------------------------------------------------------------------
\subsubsection{Geometric Representation}
\label{subsubsec:GeometricRepresentation}

\begin{figure}
\center
%	\begin{picture}(60,55)(-15,-20)
	\begin{picture}(60,65)(-15,-20)
		\setlength{\unitlength}{2pt}
%		\graphpaper(0,0)(80,70)
		\put(-15,0){\vector(1,0){50}} % x-axis
		\put(0,-15){\vector(0,1){40}} % y-axis
		\multiput(-10,-1)(10,0){5}{\line(0,1){2}} % x-axis
		\multiput(-1,-10)(0,10){4}{\line(1,0){2}} % y-axis
		\multiput(0,10)(4,0){3}{\line(1,0){2}} 
		\multiput(10,0)(0,4){3}{\line(0,1){2}} 
		\multiput(0,17)(4,0){6}{\line(1,0){2}} 
		\multiput(23,0)(0,4){4}{\line(0,1){2}} 
		\put(-1,-1){\makebox(0,0)[tr]{$O$}}	% O
		\put(10,-2){\makebox(0,0)[tc]{$1$}}	% 1
		\put(23,-2){\makebox(0,0)[tc]{$x$}}	% x
		\put(23,0){\makebox(0,0)[bl]{$X$}}	% X
		\put(-1,-1){\makebox(0,0)[tr]{$O$}}	% O
		\put(-1,-1){\makebox(0,0)[tr]{$O$}}	% O
%		\put(10,-5){\makebox(0,0)[bl]{$1$}}	% 1
%		\put(-3,10){\makebox(0,0)[bl]{$i$}}	% i
		\put(-2,10){\makebox(0,0)[cr]{$i$}}	% i
		\put(-2,17){\makebox(0,0)[cr]{$iy$}}	% iy
		\put(0,17){\makebox(0,0)[bl]{$Y$}}	% Y
		\put(23,17){\makebox(0,0)[bl]{$P$}}	% P
		\put(23,17){\makebox(0,0)[tl]{$x+iy$}}	% x+iy
		\put(10,10){\makebox(0,0)[bl]{$1+i$}}	% 1+i
	\end{picture}
	\caption{}
	\label{fig:complexRepresentationA}
\end{figure}

\begin{figure}
\center
	\begin{picture}(100,90)(-35,-40)
		\setlength{\unitlength}{2pt}
		% \graphpaper(0,0)(80,70)
		\put(-35,0){\vector(1,0){100}} % x-axis
		\put(0,-25){\vector(0,1){50}} % y-axis
		\multiput(-30,-1)(10,0){7}{\line(0,1){2}} % x-axis
		\multiput(-1,-20)(0,10){5}{\line(1,0){2}} % y-axis
		\put(65,-2){\makebox(0,0)[tl]{$X$}}
		\put(2,25){\makebox(0,0)[bl]{$iY$}}
		% 1
		\put(10,2){\makebox(0,0)[bl]{$1$}}
		% i
		\put(2,10){\makebox(0,0)[bl]{$i$}}
		% 3+2i
		\put(30,20){\makebox(0,0)[bl]{$3+2i$}}
		\multiput(0,20)(4,0){8}{\line(1,0){2}} 
		\multiput(30,0)(0,4){6}{\line(0,1){2}} 
		% 2-2i
		\put(20,-20){\makebox(0,0)[tl]{$2-2i$}}
		\multiput(0,-20)(4,0){6}{\line(1,0){2}} 
		\multiput(20,0)(0,-4){6}{\line(0,1){2}} 
		% -2+i
		\put(-20,10){\makebox(0,0)[br]{$-2+i$}}
		\multiput(0,10)(-4,0){6}{\line(1,0){2}} 
		\multiput(-20,0)(0,4){3}{\line(0,1){2}} 
		% -1-2i
		\put(-10,-20){\makebox(0,0)[tr]{$-1-2i$}}
		\multiput(0,-20)(-4,0){3}{\line(1,0){2}} 
		\multiput(-10,0)(0,-4){6}{\line(0,1){2}} 
		% z and z congugate
		\put(50,16){\makebox(0,0)[bl]{$z$}}
		\put(50,-16){\makebox(0,0)[tl]{$\overline{z}$}}
		\multiput(50,15)(0,-4){8}{\line(0,1){2}} 
%		\multiput(-10,0)(0,-4){6}{\line(0,1){2}} 
	\end{picture}
	\caption{}
	\label{fig:complexRepresentationB}
\end{figure}

By taking two perpendicular axes with a common origin $0$, 
and considering 
the horizontal axis as the \textit{real axis}
and
the vertical axis as the \textit{imaginary axis}
containing pure imaginary numbers
(See 
\reffig{fig:complexRepresentationA} and 
\reffig{fig:complexRepresentationB}),
any complex numbers $z = x + iy$ will be represented 
by a point $P$ as the vertex of the rectangle $OXPY$ 
where 
$X$ is on the real axis with abscissa $x$, and 
$Y$ is on the imaginary axis $iy$. 
The plane in which complex numbers represented this way is called \hDefined{complex plane} 
(\hDefined{z-plane} or
\hDefined{ARGAND plane}).

On the accompanying \reffig{fig:complexRepresentationB}, 
the numbers 
$1$, 
$i$, 
$3 + 2i$, 
$-2 + i$, 
$-1 - 2i$, 
$2 - 2i$ are plotted.

The conjugate numbers $z = x + iy$ and $z = x - iy$ will 
symmetrically placed with respect to real axis.

%: b1p1/014 ++++++++++++++++++++++++++++++++++++++
\hPage{b1p1/014}
% ++++++++++++++++++++++++++++++++++++++
\begin{exmp}
	$\dfrac{2 + 3i}{1 - i} 
	= \dfrac{2 + 3i}{1 - i} \dfrac{1 + i}{1 + i}
	= \dfrac{-1 + 5i}{2}
	= - \dfrac{1}{2} + \dfrac{5}{2} i$
\end{exmp}

One may show that 
\[
	\overline{z_{1} z_{2}} 
	= \overline{z_{1}} \overline{z_{2}},
	\quad 
	\overline{z_{1} / z_{2}} 
	= \overline{z_{1}} / \overline{z_{2}}
\]
In words: 
The conjugate of a product (ratio) is 
the product (ratio) of conjugates.

\begin{thm}[The Fundamental Theorem of Algebra]
	A polynomial equation with real coefficient of degree $n$ 
	has at least one root, real or imaginary, 
	and hence $n$ roots, real or imaginary, simple or repeated.

	\begin{proof}
		Omitted.
	\end{proof}
\end{thm}


\begin{cor}
	If a polynomial equation with real coefficients 
	has an imaginary root 
	it admits its conjugate as another root.
	\begin{proof}
		The proof is an applications of conjugation:
		Let the equation 
		$P(x) = a_{0} + a_{1}x + \cdots + a_{n}x^{n} = 0$ 
		which can be represented as 
		\[
			P(x) = \sum_{k = 0}^{n} a_{k} x^{k} = 0 
		\]
		admit the imaginary number $ z $ as root.Then
		\begin{align*}
			0 = P(z) 
			&= \sum a_{k} z^{k}\\
			\Longrightarrow  0 
			&= \overline{\sum a_{k}z^{k}} 
			= \sum \overline{a_{k} z^{k}} 
			= \sum \overline{a_{k}} \overline{z^{k}}\\ 
			%&%= \sum a_{k} \overline{z^{k}}
			&= \sum a_{k} (\overline{z})^{k}
			%= \sum a_{k} (\overline{z})^{k} 
			= P(\overline{z}) 
			\Longrightarrow P(\overline{z}) = 0.
		\end{align*}
	\end{proof}
\end{cor}

\begin{cor}
	A polynomial equation with real coefficients of odd degree 
	has at least one real root.
\end{cor}
Polar form of complex numbers and related properties will be treated in Chapter 4.
\footnote{@HB ref}
%: b1p1/015 ++++++++++++++++++++++++++++++++++++++
\hPage{b1p1/015}
% ++++++++++++++++++++++++++++++++++++++

We conclude this section by two classification of numbers in 
\reffig{fig:classificationOfComplexNumbersA} and 
\reffig{fig:classificationOfComplexNumbersB}.

\begin{figure}[htb]
	\centering
	\includegraphics[width=0.9\textwidth]{images/SD-1-1p15A}
	\caption{Classification of complex numbers}
	\label{fig:classificationOfComplexNumbersA}
\end{figure}
\begin{figure}[htb]
	\centering
	\includegraphics[width=0.9\textwidth]{images/SD-1-1p15B}
	\caption{Classification of complex numbers}
	\label{fig:classificationOfComplexNumbersB}
\end{figure}




% -------------------------------------------------------------------
\subsection{Exercises (\ref{sec:Numbers})}
\label{subsec:Exercises}

% +++++++++++++++++++++++++++++++++++++++++++++++++++++++++++
\begin{exercise}
\label{ex:01:001}
	Construct the following numbers on the number axis:
	\begin{center}
	\begin{tabular}{cc}
		a) 3/5
		&b) $-7/3$ (use Thales Theorem)\\
		c) $\sqrt{8}$ \quad \quad
		&$\sqrt{12}$ (use Pythagorean Theorem).
	\end{tabular}
	\end{center}
\end{exercise}
% +++++++++++++++++++++++++++++++++++++++++++++++++++++++++++
\begin{exercise}
\label{ex:01:002}
Give examples of two irrational numbers such that their
%: b1p1/016 ++++++++++++++++++++++++++++++++++++++
\hPage{b1p1/016}
% ++++++++++++++++++++++++++++++++++++++
	\begin{center}
	\begin{tabular}{llll}
		a) sum
		&b) difference
		&c) product
		&d) ratio
	\end{tabular}\\
	\end{center}
	is a rational number.
\end{exercise}
% +++++++++++++++++++++++++++++++++++++++++++++++++++++++++++
\begin{exercise}
\label{ex:01:003}
	Let $e_{1}$, $e_{2}$ be two even and 
	$o_{1}$, $o_{2}$ be two odd numbers.
	Then prove the following:\\
	a) 
	$e_{1} + e_{2}$, 
	$e_{1} e_{2}$, 
	$o_{1} + o_{2}$ are even numbers\\	
	b) 
	$e_{1} + o_{1}$, 
	$o_{1} o_{2}$
	are odd numbers
\end{exercise}
% +++++++++++++++++++++++++++++++++++++++++++++++++++++++++++
\begin{exercise}
\label{ex:01:004}
	If the product of two consecutive\\
	a) even numbers is 624,\\
	b) odd numbers is 1155\\
	find them.
	[a) $\pm 24, \pm 26$,
	b) $\pm 33, \pm35$.]
\end{exercise}
% +++++++++++++++++++++++++++++++++++++++++++++++++++++++++++
\begin{exercise}
\label{ex:01:005}
	If the sum of three consecutive
	a) integers is 294,\\
	b) even integers is 288,\\
	c) odd integers is 327\\
	find them.
	[Hint: Take the middle number as a variable.]  
\end{exercise}
% +++++++++++++++++++++++++++++++++++++++++++++++++++++++++++
\begin{exercise}
\label{ex:01:006}
	Prove that the square\\  
	a) of an even number is an even number.\\
	b) of an odd number is an odd number.
\end{exercise}
% +++++++++++++++++++++++++++++++++++++++++++++++++++++++++++
\begin{exercise}
\label{ex:01:007}
	Prove the irrationality of the numbers\\ 
	a) $\sqrt{7}$
	\quad \quad
	b) $3 + \sqrt{2}$
\end{exercise}
% +++++++++++++++++++++++++++++++++++++++++++++++++++++++++++
\begin{exercise}
\label{ex:01:008}
	Find the value of $\hAbs{2x+15}$ for  
	\begin{multicols}{2}{
	a) $x = -9$
	
	b) $x = -7,8$
	}
	\end{multicols}
\end{exercise}
% +++++++++++++++++++++++++++++++++++++++++++++++++++++++++++
%\begin{exercise}
%\label{ex:01:009}
%	Show the following properties of absolute value:
%	\begin{multicols}{2}{
%		a) $\hAbs{a}^{2} = a^{2} $\\
%		c) $\hAbs{a - b} = \hAbs{b - a}$\\
%		e) $\hAbs{a b} = \hAbs{a} \hAbs{b}$\\
%		g) $\hAbs{a + b} \leq \hAbs{a} + \hAbs{b}$
%	\columnbreak
%		
%		b) $-\hAbs{a} \leq a \leq \hAbs{a} $\\
%		d) $|a|= 0 \Leftrightarrow a = 0 $\\
%		f) $\hAbs{a / b} = \hAbs{a} / \hAbs{b}$\\
%		h) $\hAbs{\hAbs{a} - \hAbs{b}} \leq \hAbs{a - b}$
%	}
%	\end{multicols}
%\end{exercise}





% ~~~~~~~~~~~~~~~~~~~~~~~~~~~~~~~~~~~~~~~ V
\begin{exercise}
\label{ex:01:009}
	Show the following properties of absolute value:
	\begin{hbColi}{2}
		\item $\hAbs{a}^{2} = a^{2} $
		\item $\hAbs{a - b} = \hAbs{b - a}$
		\item $\hAbs{a b} = \hAbs{a} \hAbs{b}$
		\item $\hAbs{a + b} \leq \hAbs{a} + \hAbs{b}$
		\item $-\hAbs{a} \leq a \leq \hAbs{a} $
		\item $|a|= 0 \Leftrightarrow a = 0 $
		\item $\hAbs{a / b} = \hAbs{a} / \hAbs{b}$
		\item $\hAbs{\hAbs{a} - \hAbs{b}} \leq \hAbs{a - b}$
	\end{hbColi}
\end{exercise}
% ~~~~~~~~~~~~~~~~~~~~~~~~~~~~~~~~~~~~~~~ A

%: b1p1/017 ++++++++++++++++++++++++++++++++++++++
\hPage{b1p1/017}
% ++++++++++++++++++++++++++++++++++++++

% +++++++++++++++++++++++++++++++++++++++++++++++++++++++++++
\begin{exercise}
	Find the distance between the given points. 
	First express them as absolute value, and then compute.
\begin{center}
\begin{tabular}{ll}
	a) $2.72$ and $5.16$ \quad \quad b) $3.86$ and $-7.28$\\
	c) $-3.86$ and $7.28$· d) $-1.23$ and $-12.35$
\end{tabular}
\end{center}	
\end{exercise}
% +++++++++++++++++++++++++++++++++++++++++++++++++++++++++++
\begin{exercise}
	$(3 + i)^{3} = ?$ 
	\quad [Ans. $18 + 26 i$].
\end{exercise}
% +++++++++++++++++++++++++++++++++++++++++++++++++++++++++++
\begin{exercise}
	$\frac{2 + i}{3 - 2i} = ?$
	\quad [Ans. $(4 + 7 i) / 13$].
\end{exercise}
% +++++++++++++++++++++++++++++++++++++++++++++++++++++++++++
\begin{exercise}
	Write a polynomial of least degree with real coefficients 
	having the roots $3$, $1 - 2i$. 
	\quad [Ans. $x^{3}- 5 x^{2} + ll x - 15$].
\end{exercise}
% +++++++++++++++++++++++++++++++++++++++++++++++++++++++++++
\begin{exercise}
	Solve for real $x$ and $y$:
	\[
		\frac{2 - i}{3 + iy}
		= \frac{2x -3iy}{2 + i}
	\quad [\text{Ans. } x = 5/6, y = 0].
	\]
\end{exercise}
% +++++++++++++++++++++++++++++++++++++++++++++++++++++++++++
\begin{exercise}
	If $z = 5 + 4i$ find $z^{2} - 2 z + z \overline{z}$
	\quad [Ans. $60 + 32i$].
\end{exercise}




% -------------------------------------------------------------------
\section{SETS}
\label{sec:Sets}




% -------------------------------------------------------------------
\subsection{Definitions} 
\label{subsec:Definitions}

\begin{defn}
	Any collection of objects (concrete or abstract) is called 
	a \hDefined{set,} and 
	the objects in the set are its \hDefined{elements} 
	or \hDefined{members}.
	The sets are usually represented by capital letters 
	$A$, $B$, $\cdots$. 
	Two sets formed by the same elements are 
	said to be \hDefined{equal}.
\end{defn}


The set $A$ consisting of elements, say, 2, $a$, Ankara, $-7$, is denoted either by listing the elements within two braces, or by a diagram (Venn diagram) in which the elements
%: b1p1/018 ++++++++++++++++++++++++++++++++++++++
\hPage{b1p1/018}
% ++++++++++++++++++++++++++++++++++++++
are marked arbitrarily in the plane and enclosed by a closed curve:

\begin{tabular}{ll} 
	$A = \{2, a, \text{Ankara}, -7\}$
		&\includegraphics[width=0.4\textwidth]{images/SD-1-1p18-Set2Ankara}
\\
	$A = \{\text{Ankara}, 2, -7, a\}$
\end{tabular}

\begin{figure}[htb]
	\centering
	\includegraphics[width=0.4\textwidth]{images/SD-1-1p18-Set2Ankara}
	\caption{Set}
	\label{fig:Set2Ankara}
\end{figure}

The symbol 
$\in$ is used to mean "is an element of" or "belongs to", and 
$\notin$ is used otherwise. 
Then
\begin{center}
\begin{tabular}{cccc}
	$2 \in A$, \quad
	& $\text{Ankara} \in A$, \quad
	& $7 \notin A$, \quad
	& Anka $\notin A$. 
\end{tabular}
\end{center}

A set having finitely many elements is said to be a 
\hDefined{finite set}, 
and one having infinity of distinct elements an 
\hDefined{infinite set}. 
Thus 
$\{ 2, a, \text{Ankara}, -7 \}$ 
is finite, while the set 
$\{ 1, 2, 3, \cdots, n, \cdots \}$ of natural numbers is infinite.

If $S$ is a finite set, 
the number of its distinct elements is denoted by $n(S)$.
\footnote{@HB |S|?}

\begin{exmp} 
	\begin{hEnumerateArabic}
	
		\item 
		For the set $D = \{0, 1, 2, 3, 4, 5, 6, 7, 8, 9\}$ of digits 
		(\hDefined{numerals}), 
		$n(D) = 10$.
		
		\item 
		For $E = \{ \text{Venus}, \text{ Earth}, \text{ Izmir}, 
		3, \text{ Earth}, 3, -5 \}$, 
		$n(E) = 5$.
	\end{hEnumerateArabic}
\end{exmp}

Another way of representing the sets is 
by the use of a property common to all elements. 
If such a property is expressed by a true statement $p(x)$, 
then the symbol
\[
	\{ x: p(x) \} 
	\quad 
	\text{ or } 
	\quad
	\{ x \mid p(x) \}
\]
represents the set of all objects having the property $p(x)$.
%: b1p1/01 ++++++++++++++++++++++++++++++++++++++
\hPage{b1p1/019}
% ++++++++++++++++++++++++++++++++++++++
The meanings of the symbols 
$\{ x: p(x) \text{ and } q(x) \}$ 
and 
$\{ x: p(x) \text{ or } q(x) \}$ 
are clear.

\begin{exmp} ( for finite sets ):
	\begin{hEnumerateArabic}
		
		\item 
		$D 
		= \{ n: n \text{ is a digit } \} 
		= \{ 0, 1, 2, \cdots, 9 \}$
		
		\item 
		$\{ n : n \in D, \quad n \text{ is prime} \} 
		= \{ 2, 3, 5, 7 \}$
		
		\item 
		$\{ n : n \in D, \quad 1 \leq n < 7$ \} 
		= \{ 1, 2, 3, 4, 5, 6 \}
	\end{hEnumerateArabic}
\end{exmp}

\begin{exmp} 
	The following infinite sets of numbers are used 
	frequently in mathematics:
	\begin{hEnumerateArabic}
		
		\item  
		$\hSoN 
		= \{ n: n \text{ is a natural number } \} 
		= \{ 0, 1, 2, \cdots, n, \cdots \}$
		
		\item  
		$\hSoZ 
		= \{n: n \text{ is an integer } \} 
		= \{ \cdots, -2, -1, 0, 1, 2, \cdots \}$
		
		\item  
		$\hSoQ 
		= \{ r: r \text{ is a rational number } \} 
		= \{ \frac{p}{q} : p,  q \in Z, q \ne 0 \}$
		
		\item
		$\hSoQ'  
		= \{ r' : r' \text{ is an irrational number } \}$
		
		\item  
		$\hSoR 
		= \{ x: x \text{ is a real number} \} 
		= \{ x: x \in Q \text{ or } x \in Q' \}$
		
		\item  
		$\hSoC 
		= \{ z: z \text{ is a complex number } \} 
		= \{ a+ i b: a, b \in R  , i^{2}=-1\}$
	\end{hEnumerateArabic}
\end{exmp}

A  set worth of mentioning is the one having no element at all. 
It is called the \hDefined{empty set} (\hDefined{null set}) and 
denoted by \hDefinedN{$\emptyset$}, 
so that $n(\emptyset) = 0$.

\begin{exmp} 
	Each one of the following is the null set $\emptyset$:
	\begin{hEnumerateArabic}
		
		\item 
		$\{ x : x^{2} - 1 = 0, \; x \in \hSoR \}$
		
		\item 
		$\{ x : \hAbs{x} < 0, \; x \in \hSoR \}$
		
		\item 
		$\{ x : x$ is a box, $x$ is open and $x$ is closed $\}$
	\end{hEnumerateArabic}
\end{exmp}

In any particular discussion, 
a set that contains all the objects that enter into that discussion 
is called  \hDefined{the universal set}. 
%: b1p1/020 ++++++++++++++++++++++++++++++++++++++
\hPage{b1p1/020}
% ++++++++++++++++++++++++++++++++++++++
Clearly numerous universal sets exist corresponding to 
numerous particular discussions. 
A universal set is denoted by \hDefinedN{U}.

If real numbers are taken into consideration, 
$\hSoR$ is the universal set. 




% -------------------------------------------------------------------
\subsection{Subsets}
\label{subsec:Subsets} 

A set $A$ is said to be a \underline{subset} of a set $B$, 
if every element of $A$ is also an element of $B$, 
and one writes 
\[
	A \subseteq B
	\quad
	\text{(Read: $A$ is a subset of $B$)}
\]
\begin{figure}[htb]
	\centering
	\includegraphics[width=0.3\textwidth]{images/SD-1-1-SetSubset}
	\caption{$A \subseteq B$}
	\label{fig:SDb1p1SetSubset}
\end{figure}
where $B$ is said to be a \hDefined{superset} of $A$.

It follows that any set is a subset of itself, 
and we agree that the empty set is a subset of any set. 
Thus 
\[
	\emptyset 
	\subseteq \emptyset
	\subseteq \{ 1 \}
	\subseteq \{ 1 \}
	\subseteq \{ 1, 2, 3 \}
	\subseteq \hSoN 
	\subseteq \hSoZ
	\subseteq \hSoQ
	\subseteq \hSoR
	\subseteq \hSoC.
\]

If $A \subseteq B$, 
but $A \ne B$ one uses the notation
\[
	A \subset B
	\quad
	\text{(Read: $A$ is a \hDefined{proper subset} of $B$)}
\]
where $B$ contains at least one element not contained in $A$. 
With this notation the above relations can be written in the form
\[
	\emptyset 
	\subseteq \emptyset
	\subset   \{ 1 \}
	\subseteq \{ 1 \}
	\subset   \{ 1, 2, 3 \}
	\subset   \hSoN 
	\subset   \hSoZ
	\subseteq \hSoQ
	\subseteq \hSoR
	\subseteq \hSoC.
\]

\begin{exmp} 
	Write all subsets of $\{ 1, 2, 3 \}$.

	\begin{hSolution}
		$\emptyset$, 
		$\{1\}$,
		$\{2\}$,
		$\{3\}$,
		$\{1,2\}$,
		$\{1,3\}$,
		$\{2,3\}$,
		$\{1,2,3\}$.
	\end{hSolution}
\end{exmp}

If each of two sets is a subset of the other, then clearly they are equal, and vice versa:

%: b1p1/021 ++++++++++++++++++++++++++++++++++++++
\hPage{b1p1/021}
% ++++++++++++++++++++++++++++++++++++++
%: b1p1/022 ++++++++++++++++++++++++++++++++++++++
\hPage{b1p1/022}
% ++++++++++++++++++++++++++++++++++++++
%: b1p1/023 ++++++++++++++++++++++++++++++++++++++
\hPage{b1p1/023}
% ++++++++++++++++++++++++++++++++++++++
%: b1p1/024 ++++++++++++++++++++++++++++++++++++++
\hPage{b1p1/024}
% ++++++++++++++++++++++++++++++++++++++
%: b1p1/025 ++++++++++++++++++++++++++++++++++++++
\hPage{b1p1/025}
% ++++++++++++++++++++++++++++++++++++++
%: b1p1/026 ++++++++++++++++++++++++++++++++++++++
\hPage{b1p1/026}
% ++++++++++++++++++++++++++++++++++++++
%: b1p1/027 ++++++++++++++++++++++++++++++++++++++
\hPage{b1p1/027}
% ++++++++++++++++++++++++++++++++++++++
%: b1p1/028 ++++++++++++++++++++++++++++++++++++++
\hPage{b1p1/028}
% ++++++++++++++++++++++++++++++++++++++
\footnote{21-28 missing}



% -------------------------------------------------------------------
\section{Induction} 

Some theorems $p(n)$ in mathematics 
which involve the integer $n$ as a variable 
are usually proved by a method called \hDefined{induction}. 
These theorems are very often expressed 
by the use of some notations which we define below.

Let $a_{m}, \cdots, a_{i}, \cdots,  a_{n}$ be any numbers 
with $a_{i}$ as the general term 
where the integer ``$i$''  is called the 
\hDefined{index variable} or simply the 
\hDefined{index}. 
($m \leq i \leq n$)

The sum $a_{m} + \cdots + a_{i} + \cdots + a_{n}$ 
where $i$ runs from $m$ up to $n$ is denoted by 
the use of capital Greek letter $\Sigma$ (sigma) as
\[
	\sum_{i = m}^{n}  a_{i} 
	= a_{m} + \cdots + a_{n} 
	\quad 
	\text{ (summation of $a_{i}$ from $m$ to $n$) },
\] 
\hDefinedN{$\Sigma$} being called the 
\hDefined{summation notation} 
and 
the product $a_{1} \cdots a_{i} \cdots a_{n}$ is represented 
by the use capital letter $\Pi$ (pi) as
\[
	\prod_{i = m}^{n} a_{i} 
	= a_{m} \cdots a_{n}  
	\quad
	\text{ (product of $ a_{i} $ from $m$ to $n$) },
\]
\hDefinedN{$\Pi$} being called the 
\hDefined{product notation}.

\begin{exmp} \
	\begin{hEnumerateArabic}
	
		\item
		\begin{align*}
			\sum_{i = 3}^{i = 6} (2i^{2} + 5) 
			&=(2 \cdot 3^{2} + 5) 
			+ (2 \cdot 4^{2} + 5) 
			+ (2 \cdot 5^{2} + 5) 
			+ (2 \cdot 6^{2} + 5)\\
			&= 2(3^{2} + 4^{2} + 5^{2} + 6^{2}) + 4 \cdot 5 \\ 
			&= 2 \cdot 86 + 20 = 192.
		\end{align*}
	
		\item
		\begin{align*} 
			\prod_{i = 2}^{4} (2 i^{2} + 5) 
			&= (2 \cdot 2^{2} + 5)
		  	(2 \cdot 3^{2} + 5)
		  	(2 \cdot 4^{2} + 5)\\
			&=13 \cdot 23 \cdot 37.
		\end{align*}							   
	
		\item  
		\[
			\prod_{i = 1}^{n} 
			= 1 \cdots n.
		\]
	\end{hEnumerateArabic}
\end{exmp}
%: b1p1/030 ++++++++++++++++++++++++++++++++++++++
\hPage{b1p1/030}
% ++++++++++++++++++++++++++++++++++++++

The last example gives the product of all positive integers 
from $1$ up to $n$. 
This particular product is abbreviated by 
the use of notation ``$!$'' called the 
\hDefined{factorial notation}:
\[
	1 \cdots m = m! 
	\quad
	\text{(read: $m$ factorial, or factorial $m$)}
\]

Defining in addition $0!$ as $1$ we have
\[
	0! = 1, \quad 
	1! = 1, \quad
	2! = 1 \cdot 2, \quad
	3! = 1 \cdot 2 \cdot 3
\]
\[
	4! = 1 \cdot 2 \cdot 3 \cdot 4,  
	5! = 1 \cdot 2 \cdot 3 \cdot 4 \cdot 5 = 4!.5
\]
\[
	(n + 1)! = 1 \cdots n \cdot (n + 1) 
	= n! \cdot (n + 1)
\]

Another symbol is ``$\mid$'' 
which is put between two integers or between two polynomials 
to mean that the left quantity divides the right one:

\[
	5 \mid 25, \quad
	9 \mid 27, \quad
	-7 \mid 91, \quad
	x-2 \mid x^{2} - 4^{2}.
\]
Some statements to be proved by induction are the following:
\begin{align*}
	p(n):  
		&\sum_{i=1}^{n} i^{2} 
		=\frac{ n(n + 1)(2n + 1)}{6}
		\text{ for all }
		n \in \hSoN\\
	q(n):
		&n! > 2^{n} 
		\text{ for all }
		n \in \hSoN\\
	r(n):
		&x - y \mid x^{n} - y^{n}
		\text{ for all }
		n \in \hSoN_{1}
\end{align*}
where the sets 
$\hSoN_{1}$, 
$\hSoN_{4}$ or in general 
$\hSoN_{m}$ means
\[
	\hSoN_{m} 
	= \{ m, m + 1, m + 2, \cdots \}
\]
which consists of all successive integers, 
smallest of which is the integer $m \in \hSoN$.

%: b1p1/031 ++++++++++++++++++++++++++++++++++++++
\hPage{b1p1/031}
% ++++++++++++++++++++++++++++++++++++++

The proof of a theorem
\[
	``p(n), 
	\text{ for all } 
	n \in \hSoZ_{m} = \{ m, m + 1, m + 2, \cdots \}''; 
	\quad
	m \in \hSoZ
\]
by induction is done in four steps:
\begin{enumerate}

	\item 
	Verifying the truth of $p(m)$, 
	or verifying $p(n)$ for the first integer $m$ in $\hSoZ_{m}$,

	\item 
	Assuming the truth of $p(k)$ for a number $k \in \hSoZ_{m}$

	\item 
	Proving $p(k+1)$ using (2)

	\item 
	Arguing as follows:\\
	$p(m)$ is true by (1).
	Since $p(m)$ is true, 
	then $p(m+1)$ must be true by (3).
	Since $p(m+1)$ is true, 
	then $p(m+2)$ must be true again by (3).
	Continuing this way $p(n)$ must be true for all $n \in \hSoZ_{m}$
\end{enumerate}

\begin{exmp} 
	Prove by induction:
	\[
		p(n): 
		\sum_{i = 1}^{n} i^{2}
		= \frac{n(n+1)(2n+1)}{6} 
		\text{ for }
		n \in \hSoZ_{1}
	\]
\end{exmp}
\begin{proof}
	Here $\hSoZ_{m}$ is $\hSoZ_{1}$, 
	since $1$ is the least value taken by $n$.
 
\end{proof}

\begin{equation*} 
p(n):\sum^n_{(i=1)} i^2=\frac{1(1+1)(2+1)}{6} \iff 1=1 (\text{ true})  \\
\end{equation*}

(In case p(m) is false the statement is disproved and hence\\
there is no need to go further.)\\

2)Suppose $p(k)$ is true for some $k$ $\in$ $Z_1$, that is suppose\\

\begin{equation*}
\sum^k_{i=1} i^2=\frac{k(k+1)(2k+1)}{6}
\end{equation*}

%: b1p1/032 ++++++++++++++++++++++++++++++++++++++
\hPage{b1p1/032}
% ++++++++++++++++++++++++++++++++++++++





% ++++++++++++++++++++++++++++++++++++++
%: corrections
CORRECTION UP TO HERE
% ++++++++++++++++++++++++++++++++++++++
% ++++++++++++++++++++++++++++++++++++++


  \begin{enumerate}
    \item[3)] We need to prove
    \[
    p(k+1): \qquad \sum_{i=1}^{k+1} i^2 = \frac{(k+1)(k+2)(2k+3)}{6}
    \]
    under the hypothesis (2). Indeed,
    \begin{eqnarray*}
      \sum_{i=1}^{k+1} i^2 & = & \left[ \sum_{i=1}^{k} i^2 \right] + (k+1)^2 \\
      & = & \frac{k(k+1)(2k+1)}{6} + (k+1)^2 \qquad (\mbox{by (2)})\\
      & = & (k+1) \left[ \frac{k(2k+1)}{6} + k + 1 \right]\\
      & = & (k+1) \frac{k(2k+1)+6(k+1)}{6}\\
      & = & (k+1) \frac{2k^2 + 7k + 6}{6}\\
      & = & \frac{(k+1)(k+2)(2k+3)}{6}
    \end{eqnarray*}
    which is $p(k+1)$.
    \item[4)] Then $p(n)$ is true for all $n \in \mathbb{Z}_1$
  \end{enumerate}
  

  \begin{exmp}
    Prove $n!>2^n$ for all $n \in \mathbb{Z}_4$
  \end{exmp}
  \begin{proof}
    \begin{enumerate}
      \item[1)] For $m=4$, $4!>2^4$ \qquad (true).
      \item[2)] Suppose $k!>2^k$ is true for $k \in \mathbb{Z}_k$.
      \item[3)] To prove $(k+1)!>2^{k+1}$, having
      \[
      (k+1)!=k!(k+1)>2^k (k+1) \qquad (\mbox{by (2)})
      \]
    \end{enumerate}
it will suffice to show
\[
	2^k (k+1) > 2^{(k+1)}
\]
or $ k+1>2 $ which is true since k $\in \hSoZ_{4}$.

4) $a! > 2^n$ is true for all n $\in \hSoZ_{4}$.
  \end{proof}
\begin{exmp}
Prove  $ x-y|x^n-y^n $, for all n $\in \hSoZ_{1}$.
\begin{proof}
\begin{hEnumerateRoman}
\item For n=1, $ x-y|x-y $ (true)
\item Suppose $x-y|x^k-y^k$ for some k $\in \hSoZ_{1}$.

We have supposed divisibility of $x^k-y^k$ by x-y, that is, the existence
of a polynomial B(x,y) such that
\[
	x^k - y^k = B(x,y).(x-y)
\]
\item We prove $x-y|x^{k+1}-y^{k+1}$ using (ii).

To use (ii) we express $x^{k+1}-y^{k+1}$ in terms of $x^k-y^k$:
\begin{align}
x^{k+1}-y^{k+1} &= x^{k+1} - x^ky + x^ky - y^{k+1}  \\
&= x^k(x-y) + y(x^k-y^k) \\
&= x^k(x-y) + y.B(x,y)(x-y)       by(2) \\
&= [x^k + y.B(x,y)] (x-y). \\
&= C(x,y).(x-y)
\end{align}
meaning that
\[
	x-y | x^{k+1}-y^{k+1}.
\]
%: b1p1/034 ++++++++++++++++++++++++++++++++++++++
\hPage{b1p1/034}
% ++++++++++++++++++++++++++++++++++++++
\item The divisibility holds for all n $\in \hSoZ_{1}$.

\end{hEnumerateRoman}
\end{proof}
\end{exmp}

\begin{center}
EXERCISES (1.3)
\end{center}

\begin{exercise}
Evaluate

\begin{table}[ht]
\begin{tabular}{l l}
$ a) \sum_{i=2}^6 {i^2} $  & $ b) \prod_{i=2}^4 {i^2}$ \\[2ex]
$ c) \prod_{j=1}^7 {\dfrac{j}{i}} $ &  $ d) \sum_{i=2}^7 {\dfrac{j^2}{i}} $\\
\end{tabular}
\end{table}
\end{exercise}

\begin{exercise}
Write the following by the use of $\sum, \prod$ or ! .
\begin{table}[ht]
\begin{tabular}{l l l}
a) $2^2+3^2+4^2+5^2+6^2$  & b) $2^2.3^2.4^2.5^2.6^2$ & c) 3+6+9+12+15 \\[2ex]
 d) 3 . 6 . 9 . 12 . 15 & e) 5 . 10 . 15 . 20 . 25 . 30 & f) 5+10+15+20+25+30 \\
\end{tabular}
\end{table}
\end{exercise}

\begin{exercise}
Write the following int the forms (n-1)!n and (n-2)!(n-1)n .
\begin{table}[ht]
\begin{tabular}{l l l l}
a) 2! & b) 10! & c) 32! & d) 50! \\[2ex]
e) 12! & f) 100! & g) 8! & f) 5! \\
\end{tabular}
\end{table}
\end{exercise}

\begin{exercise}
The symbol $ \overline{a_n...a_0}$ represents a positive number with n+1 digits (for instance $\overline{1977}$ = 1977). A mathematician proved that the equality
\[
	\sum_{k=0}^n {a_k!} = \overline{a_n....a_0}
\]
holds only for numbers 1, 145 and 40585. Verify the equality for these numbers.
\end{exercise}

\begin{exercise}
Simplify the following
\begin{table}[ht]
\begin{tabular}{l l l l}
a) $ \dfrac{9!}{8!} $  & b) $ \dfrac{10!}{11!} $ & c) $ \dfrac{12!}{14!} $ &
d) $ \dfrac{27!}{25!} $
\end{tabular}
\end{table}
\end{exercise}
%: b1p1/036 ++++++++++++++++++++++++++++++++++++++
\hPage{b1p1/036}
% ++++++++++++++++++++++++++++++++++++++
%: b1p1/037 ++++++++++++++++++++++++++++++++++++++
\hPage{b1p1/037}
% ++++++++++++++++++++++++++++++++++++++
%: b1p1/038 ++++++++++++++++++++++++++++++++++++++
\hPage{b1p1/038}
% ++++++++++++++++++++++++++++++++++++++
%: b1p1/039 ++++++++++++++++++++++++++++++++++++++
\hPage{b1p1/039}
% ++++++++++++++++++++++++++++++++++++++
%: b1p1/040 ++++++++++++++++++++++++++++++++++++++
\hPage{b1p1/040}
% ++++++++++++++++++++++++++++++++++++++
%: b1p1/041 ++++++++++++++++++++++++++++++++++++++
\hPage{b1p1/041}
% ++++++++++++++++++++++++++++++++++++++
%: b1p1/042 ++++++++++++++++++++++++++++++++++++++
\hPage{b1p1/042}
% ++++++++++++++++++++++++++++++++++++++
%: b1p1/043 ++++++++++++++++++++++++++++++++++++++
\hPage{b1p1/043}
% ++++++++++++++++++++++++++++++++++++++
%: b1p1/044 ++++++++++++++++++++++++++++++++++++++
\hPage{b1p1/044}
% ++++++++++++++++++++++++++++++++++++++
%: b1p1/045 ++++++++++++++++++++++++++++++++++++++
\hPage{b1p1/045}
% ++++++++++++++++++++++++++++++++++++++
%: b1p1/046 ++++++++++++++++++++++++++++++++++++++
\hPage{b1p1/046}
% ++++++++++++++++++++++++++++++++++++++

\begin{tabular}{l c l}
\underline{Solution}. $ \Arrowvert {7x + 3} \Arrowvert = 5$ &  $\Rightarrow$  & $5 \leqslant 7x + 3 < 6$ \\
& $\Rightarrow$ & $2 \leqslant 7x < 3$ \; $\Rightarrow $ \; 2/7 $\leqslant x < 3/7$ 
\end{tabular} 

\newcounter{sec}
\setcounter{sec}{2} 




% -------------------------------------------------------------------
\section*{%\indent \Alph{sec}. 
\underline{TYPES OF FUNCTIONS}} 

%\newcounter{subsec}
%\setcounter{subsec}{1} \




% -------------------------------------------------------------------
\subsection*{%\indent \alph{subsec}. 
\underline{Polynomial functions}:} \

A function \\

\begin{center}
$P: \Re $ \; $\Re$, \; $P(x) = \sum_{k = 0}^{n} a_k x^k = a_n x^n + . . . + a_0$ \; $( a_i \in \Re )$
\end{center} \
is called a \underline{polynomial function} where the rule \\

\begin{center}
$a_n x^n + . . . + a_0$ 
\end{center} 
for $P$ is a polynomial of \underline{degree} $n$ (if $a_n \neq 0$). The only polynomial without degree is the \underline{zero polynomial} where all coefficients are zero.

The polynomials of degree $0$ are constant, and $P: \Re \rightarrow \Re $, $P(x) = c$ is called a \underline{constant function} whose graph is a horizontal line. A polynomial \\

\begin{center}
$P: \Re \rightarrow \Re$, \;  $P(x) = ax + b$, \; $( a \neq 0 )$
\end{center}
of degree $1$ is called a \underline{linear function} of which the particular case \\

\begin{center}
$I = \Re \rightarrow \Re$, \; $I(x) = x$
\end{center}
is called the \underline{identity function} whose graph is the line $y = x$. \\

%\setcounter{subsec}{2}




% -------------------------------------------------------------------
\subsection*{%\indent \alph{subsec}. 
\underline{Rational and irrational functions}:} \

A function 

\begin{center}
$R: \Re \rightarrow \Re$, \; $R(x) = \frac{P(x)}{Q(x)}$
\end{center}
%; b1p1/048 ++++++++++++++++++++++++++++++++++++++
\hPage{b1p1/048}
% ++++++++++++++++++++++++++++++++++++++
%: b1p1/049 ++++++++++++++++++++++++++++++++++++++
\hPage{b1p1/049}
% ++++++++++++++++++++++++++++++++++++++
\noindent %carries on from previous page
of degree $n$ in $y$, defines at most $n$ algebraic functions
which we call \underline{implicitly} \\
\underline{defined functions}. % fails to skip a line otherwise

%\begin{exmp} would go here instead of "Example.", if I could use the HBMath package.
\underline{Example.} The relation $\{(x,y) : x \in \mathbb{R}, x^2 + y^2 = 4\}$,
where $x^2 + y^2 = 4$ is of second degree in $y$, defines two
functions whose rules are obtained by solving $x^2 + y^2 = 4$
for $y$: \\ \\

\begin{figure}[htb]
\label{fig:fig49}
\begin{tabular}{c c}
$y = \sqrt{4 - y^2}$ & $y = -\sqrt{4 - y^2}$ \\
%TO DO: fix bounding box
\includegraphics[bb=0 0 415 400, scale=0.3]{images/p049_fig1.png} & \includegraphics[bb=0 0 415 400, scale=0.3]{images/p049_fig2.png} \\
Graph of the function  $y = \sqrt{4 - y^2}$ & Graph of the function $y = -\sqrt{4 - y^2}$
\end{tabular}
\end{figure}

More generally a function defined by a relation $f(x, y) = 0$ is said to be an
\underline{implicitly defined function.} For instance $xy^2 - (x+1)y + 1 = 0$, 
$y$ cos $y + x^3 + x = 0$ define some implicitly defined functions. \\

d. \underline{Trigonometric Functions.}

A function which is not algebraic is called a \underline{transcendental function.} 
As some examples for transcendental functions we will give trigonometric functions
which we will represent simply by their rules: \\ \\

\begin{tabular}{c c c c}
\underline{Rules for trig. fn.} & \underline{Domain} & \underline{Range} & \underline{Period = T} \\
$y =$ sin $x$ & $\mathbb{R}$ & $[-1,1]$ & 2$\pi$ \\
$y =$ cos $x$ & $\mathbb{R}$ & $[-1,1]$ & 2$\pi$ \\
$y =$ tan $x$ & $\mathbb{R} - \{x : x = (2k + 1)\frac{\pi}{2}, k \in \mathbb{Z}\}$ & $\mathbb{R}$ & $\pi$ \\
$y =$ cot $x$ & $\mathbb{R} - \{x : x = k\pi, k \in \mathbb{Z}\}$ & $\mathbb{R}$ & $\pi$ \\
$y =$ sec $x$ & $D_{tan}$ & $\mathbb{R} - (-1,1)$ & 2$\pi$ \\
\end{tabular}
%: b1p1/050 ++++++++++++++++++++++++++++++++++++++
\hPage{b1p1/050}
% ++++++++++++++++++++++++++++++++++++++

\begin{tabular}{llll}
$y = \csc x $ &  $D_{\cot x}$ & $R - (-1, 1)$ & $2\pi$  \\
\end{tabular}
\\
\\
their graphs are given in an interval of length T: \\

\begin{figure}[h]
\begin{tabular}{ccc}
	\includegraphics[bb=0 0 415 400, scale=0.3]{images/p50_fig1.png} 
	&\includegraphics[bb=0 0 415 400, scale=0.3]{images/p50_fig2.png} 
	&\includegraphics[bb=0 0 415 400, scale=0.3]{images/p50_fig3.png} \\
\end{tabular}
\end{figure}


\underline{Identities:} \\
\begin{gather*} 
	\cos^2 x + \sin^2 x = 1, \;\; 1 + \tan^2x = \sec^2x, \;\; 1 + \cot^2x = \csc^2x \\
	\sin(x \pm y) = \sin x \cos y \pm \cos x \sin y \\
	\cos(x \pm y) = \cos x \cos y \mp \sin x \sin y \\
	\tan(x \pm y) = \frac{\tan x \pm \tan y}{1 \mp \tan x \tan y}
\end{gather*}
\begin{equation*}
\left. \begin{array}{l}
\sin 2x = 2 \sin x \cos x \\ \\
\cos 2x = \cos^2 x - \sin^2 x = 2 \cos^2 x - 1 = 1 - 2 \sin^2 x \\ \\
\tan 2x = \frac{2 \tan x }{1 - \tan^2 x}
\end{array} \right\} \text{Double Angle Formulas}
\end{equation*}
\\
\begin{equation*}
\left. \begin{array}{l}
\sin^2 \frac{x}{2} = \frac{1 - \cos x}{2} \\ \\
\cos^2 \frac{x}{2} = \frac{1 + \cos x}{2}
\end{array} \right\} \text{Half Angle Formulas}
\end{equation*}

\[ \left. \begin{array}{ll}
sinx+siny=2sin\frac{x+y}{2}cos\frac{x-y}{2}\\

\\
sinx-siny=2sin\frac{x-y}{2}cos\frac{x+y}{2}\\

\\
cosx+cosy=2cos\frac{x+y}{2}cos\frac{x-y}{2}\\

\\
cosx-cosy=-2sin\frac{x+y}{2}sin\frac{x-y}{2}\\
\end{array} \right  \}   (Factor form) 
\] 
\begin{tabbing}
\ \ \ \ \ \ \ \ \ \ \
\=C. \underline{Monotonic increasing (decreasing) functions:}\\
\\
\>A function f : D $\rightarrow$ R is said to be an \underline{increasing func}- \\
\\
\underline{tion} on an open  interval I which is a subset of the domain D, if\\
\\
\ \ \ \ \ \ \ \ \ \ \ \ \ \ \ \ \ \ \ \
$f(x_{2})>f(x_{1})$ or $f(x_{2})-f(x_{1})>0$\\
\\
for any \= two numbers $x_{1}$, $x_{2}\epsilon I$ for which $x_{1}<x_{2}$.\\
\\
\>The graph of such a function rises as x increases on \\
\\
I, and we say that f \underline{increases} on I.\\
\\
\>Under the same condition for $x_{1},$ $x_{2}$ if\\
\\
\ \ \ \ \ \ \ \ \ \ \ \ \ \ \ \ \ \ \ \ 
$f(x_{2})<f(x_{1})$ or $f(x_{2})-f(x_{1})<0,$\\
\\
than f i\=s called a \underline{decreasing function} on I.\\
\\
\>The graph of a decreasing function falls as x increases\\
\\
on I, and we say that f \underline{decreases} on I.\\
\\
\>\underline{Example}. Show that $y = 4 - x^{2}$ increases on the interval\\
\\
$R_{0}^{-}$. and decreases on  $R_{0}^{+}.$\\
\\
\>\underline{Solution}. For $x_{1},$ $x_{2}\epsilon D = R$ with $x_{1}<x_{2},$ we have\\
\\
\ \ \ \ \ \ \ \ \ \ \ \ \ \ \ \ \ \ \ \ 
$f(x_{2})-f(x_{1}) = (4 - x_{2}^{2}) - (4 - x_{1}^{2}) = x_{1}^{2} - x_{2}^{2}$\\

\end{tabbing}

%: b1p1/052 ++++++++++++++++++++++++++++++++++++++
\hPage{b1p1/052}
% ++++++++++++++++++++++++++++++++++++++
\[ \left. \begin{array}{ll}
=(x_{1}-x_{2})(x_{1}+x_{2})\\
\end{array} \right \{
\begin{array}{ll}
>0   when  x_{1}, x_{2}\epsilon R_{o}^{-}\\

<0 when x_{1}, x_{2}\epsilon R_{o}^{+}
\end{array} \]

If f is an increasing (or decreasing) function on an  interval $I\subseteq D,$ then f is said to be a \underline{monotonic increasing} (or \underline{monotonic decreasing}) function in the interval I.


The function given in the above example, is monotonic increasing in $R_{o}^{-}$ and monotonic decreasing in $R_{o}^{+}$.


A monotonic increasing (or decreasing) function f an interval is expressed usually by saying that f is \underline{one-to-} \underline{one} (or simply \underline{1-1}) in I to mean that to distinct numbers $x_{1},x_{2}$ in I correspond distinct images $f(x_{1}), f(x_{2})$.\\
\\

D. \underline{Inverse of a function}\\

A function
\begin{equation}
f:D \rightarrow \mathbb{R}, y=f(x) or f=\{(x,y): x \in D, y=f(x)\} 
\end{equation}
with D as the domain and R as the range, being a relation from D $\rightarrow$ R, its inverse
\begin{equation}
f^{-1} = \{(x,y): x \in R, x=f(y)\}
\end{equation}

is a relation from R to D.If the relation $f^{-1}$ is function we call $f^{-1}$ the \underline{inverse function} of $f$, and $f$ is said to be an \underline{invertible} on the set D.

Since $f$ is a function it maps an $x$ in D into a image $y$ in R, and since $f^{-1}$ is a function from R to D it maps $y$ backward to the single image $x$ in D. This means that $f$ is an one-to-one function and consequently $f^{-1}$ is one-to-one function.
 
 The graphs of $f$ and $f^{-1}$ are symmetric with respect\\

to the line y=x.( The pairs (x,y) of f and (y,x) of $f^{-1}$ are symmetrical in y=x )\\

$\underline{Example}$.  Show that f:R$\rightarrow$ R,y=2x-1 is invertible on R. and find its inverse g.

       \begin{center}
        f=\{(x,y): x$\epsilon$R, y=2x-1 \}\\

        $f^{-1}$ =\{(x,y): x$\epsilon$R , x=2y-1 \}\\

         =\{(x,y): x$\epsilon R$, y= $\frac{x+1}{2}$ \}\\

         g:R$\rightarrow$R, g(x)=$\frac{x+1}{2}$\\
       \end{center}


\underline{Corollary.If f:  $D\rightarrow R$ , y=f(x) is monotone increasing (or decreasing ) on} \\
\underline{a interval $I \subseteq D$,then f is invertible on that interval I.}

\begin{minipage}[t]{0.6\textwidth}
\vspace{0pt}
    $\underline{Proof}$ . It will suffice to give the proof for the case where f is monotone increasing on I.\\

    Since f is monotone  increasing it maps distinct numbers in I to distinct numbers in R.\\

    If the relation $f^{-1}$ is not a function then some distinct numbers y1,y2 $\epsilon$ R are mapped to the same number x in D,
    contradicting that f is monotone on I.\\
\end{minipage}
\hfill
\begin{figure}[htb]
\begin{minipage}[t]{0.8\textwidth}
\centering\vspace{0pt}

%\includegraphics[width=5cm]{1-1-053-054-1.png}
\caption{Missing Figure: Pg. 53}
\end{minipage}

\end{figure}

    Let f:R$\rightarrow$R be a function with a domain $D\subseteq R$. If I in a subset of D, then f:$I\rightarrow J$ is said to be a
    \underline{ restricted  function } in the restricted domain I.\\

    If there are some intervals on which a function f satisfies required conditions, then  f is  said to be restricted on each
    interval or a subset of it, and the interval itself is the largest.\\

%: b1p1/054 ++++++++++++++++++++++++++++++++++++++
\hPage{b1p1/054}
% ++++++++++++++++++++++++++++++++++++++

    $\underline{Example}$.Find a restriction on the domain D of the function given by the rule y= $\mid x-1 \mid -2 \mid x \mid + x$
    to be

     a) a constant function,

     b) an inveritble  function.

    $\underline{Solution}$. The given function is the piecewisely defined function:


\begin{center}
   $$
\mbox{y}=\left\{
\begin{array}{rl}
1 + 2x & \mbox{if $x \epsilon (-\infty,0)$} \\
1 - 2x & \mbox{if $x \epsilon (0,1]$} \\
-1 & \mbox{if x (1,$\infty$).}
\end{array} \right.
$$
\end{center}


    a) A domain of restriction is (1,$\infty$).

    b) A domain of restriction is (-$\infty$,0] on which the function is increasing or (0,1] on which it is decreasing.

    E.\underline{Operation with functions}:

    Let

    \begin{center}
    f:$I\rightarrow R$, y=f(x)
    \end{center}


be a function with domain I.If c$\epsilon$R, then the function

    \begin{center}
    cf: $I \rightarrow R$, y=(cf)(x)=cf(x)   (0)
    \end{center}


is called a \underline{scalar multiple} of f.

    Let now be given two functions

\begin{center}

    f:$I\rightarrow R$, y=f(x)

    g:$J\rightarrow R$, Y=g(x)
\end{center}

with non disjoint domain I and J, then f+ g,f-g, fg ,


f/g called the $\underline{sum}$, $\underline{difference}$ , $\underline{product}$  and  $\underline{ratio}$  of f and g, are defined as follows:\\

\hspace{10mm} $\underline{Domain}$\\
f+g: $I\cap J$, y=(f+g)(x) = f(x)+g(x)   $\hspace{3mm}$ (1)\\
f-g: $I\cap I$, y=(f-g)(x) = f(x)-g(x)   $\hspace{3mm}$ (2)\\
f*g: $I\cap J$, y=(fg)(x) = f(x)g(x)   $\hspace{3mm}$ (3)\\
f/g: $D$, y=(f/g)(x) = f(x)/g(x)   $\hspace{3mm}$ (4) $\hspace{3mm}$
where D = $(I\cap J) - {x: g(x)= 0}$.\\
Another function is gof, called $\underline{composite function}$ which is defined as \\
gof:D, v=(gof)(x) = g(f(x))\\
where the domain D is the largest possible subset of $\Re$ on which g(f(x)), f(x) and g(x) are defined.\\
Because of the rule g(f(x)) we call also a $\underline{function of function}$ or a $\underline{chain function}$.\\
$\underline{Example}$. let f(x)= $\frac{|x|}{x}$ and g(x) = x$\sqrt{(1-x)}$ be two functions. We have\\
D$_{f}$ = $R^*$ = R ${0}$,  D$_{g}$ $=(-\infty, 1]$\\
and\\
0) $ 3f(x) =3f(x) =  3\frac{|x|}{x} $\\ \vspace{4 mm}
1) $  (f+g)(x) = f(x)+g(x)  = \frac{|x|}{x} + x\sqrt{(1-x)} $\\ \vspace{4 mm}
2). $ (f-g)(x) = f(x)-g(x) = \frac{|x|}{x} - x\sqrt{(1-x)} $\\ \vspace{4 mm}
3). $ (fg)(x) = f(x)g(x) = \frac{|x|}{x} x\sqrt{(1-x)} = |x|\sqrt{(1-x)} $\\ 
%: b1p1/056 ++++++++++++++++++++++++++++++++++++++
\hPage{b1p1/056}
% ++++++++++++++++++++++++++++++++++++++
where cancelation by x is permissible under x$/neq$ and this condition is jointly written with the rule.
%\newpage
%\begin{center}
%     \Large{\textbf{55}} \\
%\vspace{10 mm}
%\end{center}
4) $ (\frac{f}{g})(x) = \frac{f(x)}{g(x)}) = \frac{|x|}{x^2\sqrt{(1-x)} } $\\
As to compositions gof and fog we have\\
\vspace{4 mm}
5) $ (gof)(x) = f(g(x)) = g(\frac{|x|}{x}) =\frac{|x|}{x} \sqrt{1-\frac{|x|}{x}}$\\
$fog(x) = f(g(x))= f(x\sqrt {(1-x)}) = \frac {|x\sqrt{ (1-x)| }} {x\sqrt {(1-x)}} = \frac {|x|\sqrt {(1-x)}} {x\sqrt {(1-x)}}$\\
\begin{center}
    $ = \frac {|x|}{x}$ \hspace{3mm} ($x\neq 1$)\\
\end{center}
and\\
D$_{gof}$=$(-\infty , 1 ] - {0}$ , D$_{fog}$ = $(-\infty , 1 ] - {0,1}$  = $(-\infty , 1 ) - {0}$ = $(-\infty , 1 )^*$\\
$\underline{Example}$. Given the functions \\
f:$\Re \rightarrow \Re $ \hspace{2mm} f(x)= $\frac{x}{x-2}$ ; g:$\Re \rightarrow \Re $ , g(x)= $x^2-x$\\
find the rules for composite functions gof fog, and then determine their domains.
$\underline{Solution}$. 
1.$(gof)(x) = g(f(x)) = f^2(x)- f(x) = \frac{x^2}{(x-2)^2}- \frac{x}{x-2} =\frac{ x^2 - x(x-2)}{(x-2)^2}  = \frac {2x} {(x-2)^2}  $\\
\vspace{2 mm}
2.$(fog)(x) = f(g(x)) = \frac{g(x)}{g(x)-2} = \frac{x(x-1)}{(x+1)(x-2)}    $\\
$D_{gof} = \Re - {2} $ , \hspace{4mm} $D_{fog}= \Re -{ -1 ,2 }$\\

\textbf{\underline{Corollary:}} If f is an invertible function, then
\begin{equation*}
f^{-1} \circ f  =  f \circ f^{-1}  = I 
\end{equation*}
where $ I $ is the identity function under a necessity restriction.\\

\textbf{\underline{Proof:}} Let $f: D \rightarrow  R$  $y = f(x) $
with $x = f^{-1}(y)$ then 
\begin{equation*}
( f^{-1} \circ f)(x) = f^{-1}(f(x)) = f^{-1}(y) = x = I(x)
\end{equation*}
$(f^{-1} \circ f)(x) = I(x) $ for all $ x $ implying that $f \circ f^{-1} = I $

Also
\begin{equation*}
(f \circ f^{-1}) (y) = f(f^{-1}(y)) = f(x) = y = I(y) \Rightarrow f \circ f^ {-1} = I ~ \Box
\end{equation*}

\textbf{\underline{Corollary:}} $(h \circ g)\circ f  = h \circ (g \circ f)$
\\
\begin{align*}
For, ((h \circ g)\circ f)(x) &= (h \circ g)(f(x))\\
&=h(g(f(x))) = h \circ ((g \circ f)(x)) = (h \circ (g \circ f))(x)
\end{align*}
\\for all x.~$\Box$

\textbf{\underline{Corollary:}} If $ f$ and $g $ are invertible functions, then 
\begin{equation*}
(g \circ f)^{-1} = f^{-1} \circ g^{-1}
\end{equation*}

\textbf{\underline{Proof:}} We need to show that $(g \circ f)\circ(f^{-1} \circ g^{-1}) = I $
\\Indeed\\
\begin{align*}
(g \circ f) \circ (f^{-1} \circ g^{-1}) &= g \circ(f \circ f^{-1}) \circ g^{-1}\\
 &= g \circ I \circ g^{-1} = g \circ (I \circ g^{-1}) = g \circ g^{-1} =  I ~ \Box
\end{align*}
%: b1p1/058 ++++++++++++++++++++++++++++++++++++++
\hPage{b1p1/058}
% ++++++++++++++++++++++++++++++++++++++
F.\underline{Even and odd functions}\\
\\
Let $f:$ be $D \rightarrow \mathbb{R}$ be a function with $x\in D \Rightarrow -x\in D$.Then\\
f is called\\
\\
	1$)$an \underline{even} function if $f(-x)=f(x)$ for all $x\in D$,\\
	2$)$an \underline{odd} function if $f(-x)=-f(x)$ for all $x\in D$.\\
\begin{exmp} 
for $n\in N$\\
 \\
	1$)$ $f(x)=x^{2n}$ is an even function.\\
	2$)$ $f(x)=x^{2n+1}$ is an odd function.\\
\end{exmp}

Solution.\\
	1$)$ $f(-x)=(-x)^{2n}=x^{2n}=f(x)$ for all $x\in \mathbb{R} $\\
	2$)$ $f(-x)=(-x)^{2n}=-x^{2n}=-f(x)$ for all $x\in \mathbb{R} $\\
\\
	The reader can show that the function $ f(x)=x^{3}-x^{2} $\\
is neither even nor odd, and that the zero function $ 0(x)= 0 $\\
is both even and odd.\\
	Why the graph of an even (odd) function is sym. w.r.to\\
y-axis (origin)?\\
\\
G.\underline{Periodic Functions}\\
\\
A function $f:\mathbb{R}\rightarrow \mathbb{R}$ with domain $\mathbb{R}$ is said to be\\
\underline{periodic} if there exist a number $T(\neq 0)$ such that\\
\\
		$f(x + T)=f(x)$ for all $x\in \mathbb{R} $\\
\\
where T is called \underline{a period} of x.\\
If T is a period, certainly, all integral multiples\\
of T are also periods.\\
	The smallest of all positive periods is called the \underline{funda}-\\
\underline{mental period} or \underline{the least period} or \underline{the period} of f, written\\
$T_{f}$. As a period of constant function may be taken any real\\
number.\\




% -------------------------------------------------------------------
\subsection{\emph{Examples}}

\textbf{1}.sin(x),cos(x)  ($T_{F}=(2\pi ));$

\bigskip 

\textbf{2}.tan(x) , cot(x) ($T_{F}=(\pi ));$

\bigskip 

\textbf{3.}x - [x], $\ \ \ \ \ \ \ \ \ \ \ (\ \ \ T_{F}=1);$

\bigskip 

The graph of a perioadic function is obtained with the repetition of the
graph of f in the interval of length $T_{F}.$

\bigskip 




% -------------------------------------------------------------------
\subsection{\emph{Corrolaries}}

\bigskip 

\textbf{1}.f(x +t) = f(x) $\Rightarrow$ f(x +a + t)=f(x + a)

\bigskip 

\textbf{2.} $T_{cf}=T_{f}$ \ (c$\in R)$

\bigskip 

\textbf{3}. If the period of f is $T_{f}$ ,then the period of f(ax + b) is $%
T_{f}/a$:  Suppose f(ax +b) is periodic with period T' . Then 

\bigskip 

f(a(x +T') + b) = f(ax + b) holds implying

\bigskip 

f(ax + b+ aT') = f(ax + b) $\Rightarrow$ aT'=$T_{f}=>T^{\prime }=T_{f}/a.$

\begin{flushleft}
\textbf{Example.}Find the periods of cos(3x + 2) and tan ($\frac{x}{5}).$

\bigskip 

\textbf{Answer. }$\frac{2\pi }{5},5\pi $

\bigskip 
\end{flushleft}

\textbf{4.}If the periods of f , g are $T_{f},T_{g}$ ,respectively,

\bigskip 

then f +g ,f-g, fg ,f/g are periodic and positive

\bigskip 

period T is in interval of length T such that 

\bigskip 

T/$T_{f},T/T_{g}$ are positive integers . \ 

\bigskip 

\begin{flushleft}
\textbf{Example.} Find a period of cos(x) + cos(3x)

\bigskip 

\textbf{Solution.}Let f(x) =cosx and g(x) = cos 3x. Then we have T$_{f}=2\pi 
$ \ and T$_{g}=2\pi /3$ implying that T=2$\pi $ since T/T$_{f}=1,$ T/T$%
_{g}=3.$

\bigskip 

\textbf{Example.}Find a period of 2 sin(x) cos(x).

\bigskip 

\textbf{Solution.}Period of sin(x),cos(x) being 2$\pi ,2\pi ,a$
\end{flushleft}


%: b1p1/060 ++++++++++++++++++++++++++++++++++++++
\hPage{b1p1/060}
% ++++++++++++++++++++++++++++++++++++++
%: b1p1/061 ++++++++++++++++++++++++++++++++++++++
\hPage{b1p1/061}
% ++++++++++++++++++++++++++++++++++++++


period is $T= 2 \pi$ , but this not the least period, because $2\sin x\cos x = \sin 2x$ has period $2\pi /2=\pi$ .
\textbf{5.}$gof$ is periodic if f is periodic:
	
	$(gof)(x+ T_{f})=g(f(x+ T{f}))=g(f(x))=(gof)(x)$.
\textbf{\underline{Inverse Trigonometric Functions}}
Each of the six trigonometric functions has an inverse in an interval in which it is increasing or decreasing. For each one, a fundamental restricted interval is selected. This interval for a particular function will be the fundamental range of the inverse of that function.

\begin{center}
Trigonometric functions,\\
their intervals of increase or decrease,\\
and chosen fundamental intervals 
\end{center}

\begin{quote}
\begin{tabular}{l l l}
f&Intervals of increase or decrease of f & fundamental interval\\
\hline
$y= \sin x$&$[(2k-1)\frac{\pi}{2}, (2k+1)\frac{\pi}{2}]$ &$[-\frac{\pi}{2},\frac{\pi}{2}]$\\
$v= \cos x$&$[k\pi,(k+1)\pi]$&$[0,\pi]$ \\
$y=\tan x$&$((2k-1)\frac{\pi}{2}, (2k+1)\frac{\pi}{2})$&$(-\frac{\pi}{2},\frac{\pi}{2})$\\
$y=\cot x$&$(k\pi,(k+1)\pi)$&$(0,\pi)$ \\
$y=\csc x$&$((2k-1)\frac{\pi}{2}, (2k+1)\frac{\pi}{2})$&$(-\frac{\pi}{2},\frac{\pi}{2})$\\
$y=\sec x$&$(k\pi,(k+1)\pi)$&$(0,\pi)$ \\
\end{tabular}
\end{quote}
%: b1p1/062 ++++++++++++++++++++++++++++++++++++++
\hPage{b1p1/062}
% ++++++++++++++++++++++++++++++++++++++
%: b1p1/063 ++++++++++++++++++++++++++++++++++++++
\hPage{b1p1/063}
% ++++++++++++++++++++++++++++++++++++++
%: b1p1/064 ++++++++++++++++++++++++++++++++++++++
\hPage{b1p1/064}
% ++++++++++++++++++++++++++++++++++++++
%: b1p1/065 ++++++++++++++++++++++++++++++++++++++
\hPage{b1p1/065}
% ++++++++++++++++++++++++++++++++++++++
%: b1p1/066 ++++++++++++++++++++++++++++++++++++++
\hPage{b1p1/066}
% ++++++++++++++++++++++++++++++++++++++
\begin{tabular}{c c c c}
a) $x$ & b) $|x|$ & c) $x^2 - 1$ & d) $4 - x^2$ \\
e) $\cos x$ & f) $\sin x$ &(for $(e)$,$(f)$, $x \in \left[ 0,2\pi \right]$)\\
\end{tabular}

\begin{enumerate}
\item Prove
\begin{enumerate}
\item[a)] $(f+g)\circ h = f\circ h + g\circ h$
\item[b)] $(f-g)\circ h = f\circ h - g\circ h$
\item[c)] $(fg)\circ h = (f\circ h)(g\circ h)$
\item[d)] $(f/g)\circ h = (f\circ h)/(g\circ h)$
\end{enumerate}
\item Write the intervals in which the following functions are monotone(you may use graph):\\
\begin{tabular}{c c c}
a) $ y = \frac{1}{x+3} $ & b) $ y = \sin x + \cos x $ &c) $ y = |x^2-4|+4$ \\ 
\end{tabular}
\item Find the inverse of the function given in Exercise 74 choosing one proper interval.
\item Find the inverse of the function
\begin{displaymath}
f(x) = \left\{
\begin{array}{lr}
3x - 1 $ $ when &  x \leq -1 \\
\frac{3x}{x+2}$ $  when  & x > -1
\end{array}
\right.
\end{displaymath}
\item Find the points of intersection, if any, of the given pairs of functions:
\begin{enumerate}
\item[a)] $y = \frac{x+2}{x-1}$, $y = \frac{x-2}{x+1}$
\item[b)] $y = \frac{2x-1}{x+3}$, $y = \frac{3x+1}{2-x}$
\end{enumerate}
\item If $f(x) = \sin x$ and $g(x) = x^2 + 2$, then find \\ 
\begin{tabular}{c c}
a) $ f(\frac{x}{2} + \pi)g(2x-1)$ & b)$f(3a)g(\sin a)$ \\ 
\end{tabular}
\item Find the ranges of the following functions(Hint: Solve for x!) 
\end{enumerate}


\begin{multicols}{2}
\begin{hItemizeBullet}
  \item $y = \frac{x^2 - 3x}{x+1}$
  \item $y = \frac{x^2}{x^2 - 2x -3}$ 
\end{hItemizeBullet}
\end{multicols}


\begin{hQuestion} {Find the periods of}

\begin{multicols}{3}
\begin{hEnumerateRoman}

  \item $\cos \hPairingParan{2x + 3}$
  \item $\sin \hPairingParan{\frac{x}{3} - 2}$
  \item $\tan \hPairingParan{\frac{x}{2} + \Pi }$
  \item $\cot \hPairingParan{3x - \Pi }$
  \item $\cos \hPairingParan{\Pi x - \Pi }$ 
  \item $\sin \hPairingParan{2 \Pi x - \Pi ^2 }$
  \item $\sin x \cos x $
  \item $\tan ^2 x$

\end{hEnumerateRoman}
\end{multicols}
\end{hQuestion}


\begin{hQuestion} {Examine the following functions for evenness and oddness}

\begin{multicols}{4}
\begin{hEnumerateRoman}

  \item $\mid x \mid$
  \item $ 3-x $
  \item $ x + 2 x ^3 $
  \item $ x \mid x \mid $
  \item $ \mid x \mid - x^2 $
  \item $ -3 $
  \item $ \sin ^3 2x $
  \item $ \frac{\sin 2x}{\sin 3x} $
  
\end{hEnumerateRoman}
\end{multicols}

\end{hQuestion}


\begin{hQuestion} {Find $f\circ g$ and $g \circ f $} if

\[
f(x) = \sqrt{x + 1} 
\]
\[
and
\]
\[
 g(x) = \frac{x}{x^2 - 4x +3}
\]

and determine the domain of each of these composite functions.

\end{hQuestion}


\begin{hQuestion} {Express the area of }

\begin{figure}[h]
\center
\includegraphics[scale=.5, bb = 0 0 450 330]{images/p067_fig1.png}
\caption{ AOB Triangle and ACOD Rectangle}
\label{fig:p067_fig1}
 
\end{figure}

\begin{hEnumerateRoman}

  \item the triangle AOB in terms of $\Theta$
  \item the triangle AOB in terms of x
  \item the rectangle ACOD in terms of $\Theta$
  \item the rectangle ACOD in terms of x

\end{hEnumerateRoman} 

\end{hQuestion}


\begin{hQuestion} {Find the domain of restriction in which the relation}

	$\mid x + y \mid - y + 2  = 0 $ is a function. 

\end{hQuestion}


\begin{hQuestion} {Given the relation $9x^2 - 36x +16y^2 +96y+36 =0$ .}
	Write two functions equivalent to this relation.
\end{hQuestion}
%: b1p1/068 ++++++++++++++++++++++++++++++++++++++
\hPage{b1p1/068}
% ++++++++++++++++++++++++++++++++++++++
Answers to even numbered exercises

\begin{flushleft}



 56.Only a). 58. Polynomials: F,H; Rational functions: f,g,F,H; Irrational 

function:h,i; Algebraic functions: f,g,h,F,G,H; Trans. functions: i.

60. a) R - \{-1, 1\}, R - \{0, 1/2\}.

 b) [1, 2 ], [0, $\infty$) .
 
 c)R - [2, 4], R - [2 , 16] .
 
62. y = 1/x , y = x .

64. a) [2k$\pi$-$\pi$/2 , 2k$\pi$+$\pi$/2], increasing; [2k$\pi$+$\pi$/2 , 2k$\pi$+3$\pi$/2], decreasing.
	
	b)[k$\pi$-$\pi$/2 , k$\pi$+$\pi$/2], increasing.
	
	c)[2k$\pi$ , (2k+1)$\pi$] increasing; [(2k+1)$\pi$ , (2k + 2)$\pi$] decreasing. 
	
	d)[k$\pi$ , k$\pi$+$\pi$/2] decreasing; [k$\pi$+$\pi$/2 , k$\pi$+$\pi$] increasing.
	
66. Missing figure

70. a) y = x + 5  b) y = -1  c) x = 3, not a function.

d)$x = y^2 - 1$, not a function e) y = cos x f) y = arcsin x

72. Missing figure

74.a) (-$\infty$, -3) , (-3, $\infty$ ) b) [3$\pi$/4, 5$\pi$/4] , [5$\pi$/4, 7$\pi$/4]

c) (-$\infty$, -2) , [-2, 0] , [0, 2] , [2, $\infty$).

\end{flushleft}
%: b1p1/073 ++++++++++++++++++++++++++++++++++++++
\hPage{b1p1/073}
% ++++++++++++++++++++++++++++++++++++++

\underline{Left and right limits:}\\
\indent The limit of a function f at a point $x_0$ under the \\
conditions\\
$$x< x_0 \quad , \quad 0<|x-x_0|<\delta  $$ 

is called the \underline{left limit} of f at $x_0$, and the limit of f at \\
$x_0$ under the conditions\\
$$x> x_0 \quad , \quad 0<|x-x_0|<\delta  $$ 

is called the \underline{right limit} of f at $x_0$.\\
\indent \indent The notations for left limit are\\\\

$\lim_{\substack{x\to x_0 \\ x<x_0}}f(x)$, \indent  $\lim_{\substack{x\uparrow x_0}}f(x)$,  \indent $\lim_{\substack{x\nearrow x_0}}f(x)$, \indent $\lim_{\substack{x\to x_0^-}}f(x)$,
\indent $\lim_{\substack{x\to x_0-}}f(x)$ \\\\
and those for right one are:\\\\
$\lim_{\substack{x\to x_0 \\ x>x_0}}f(x)$, \indent  $\lim_{\substack{x\downarrow x_0}}f(x)$,  \indent $\lim_{\substack{x\searrow x_0}}f(x)$, \indent $\lim_{\substack{x\to x_0^+}}f(x)$,
\indent $\lim_{\substack{x\to x_0+}}f(x)$ \\\\


At a given point $x_0$ some functions have both the left and right limit, some others have only one, and still others have none.

If both the left and right limit exist at $x_0$ for a function f, and are equal to each other (=$\ell$), then we say  that f(x) has the limit $\ell$, and one writes
 \begin{center}
$\lim_{\substack{x\to x_0}}f(x) = \ell$
\end{center}
\indent \indent If f: I $\rightarrow$ R, where I is an interval with end points a
%: b1p1/078 ++++++++++++++++++++++++++++++++++++++
\hPage{b1p1/078}
% ++++++++++++++++++++++++++++++++++++++
\begin{align}
&\left \lvert f(x)- \ell \right \rvert < \varepsilon ( \varepsilon < \left \lvert \ell \right \rvert  is \,\, taken )\\
\Longrightarrow &\left \lvert \left \lvert f(x) \right \rvert - \left \lvert \ell \right \rvert \right \rvert \leq \left \lvert f(x)- \ell \right \rvert < \varepsilon (  From \left \lvert \left \lvert a \right \rvert - \left \lvert b \right \rvert \right \rvert \leq \left \lvert a - b \right \rvert )\\
\Longrightarrow &\left \lvert \left \lvert f(x) \right \rvert - \left \lvert \ell \right \rvert \right \rvert < \varepsilon\\
\Longrightarrow &-\varepsilon < \left \lvert f(x) \right \rvert - \left \lvert \ell \right \rvert < \varepsilon \\
\Longrightarrow &0 <  \left \lvert \ell \right \rvert - \varepsilon < \left \lvert f(x) \right \rvert <  \left \lvert \ell \right \rvert + \varepsilon \indent ...(i)
\end{align}
Now for $x \in N(x_{0})$\\

$\left \lvert \dfrac{1}{f(x)} - \dfrac{1}{\ell}  \right \rvert = \dfrac{\left \lvert f(x) - \ell \right \rvert}{\left \lvert f(x) \right \rvert \left \lvert \ell \right \rvert} < \dfrac{\varepsilon}{\left \lvert f(x) \right \rvert \left \lvert \ell \right \rvert} < \dfrac{\varepsilon}{(\left \lvert \ell \right \rvert - \varepsilon)\left \lvert \ell \right \rvert} $\\

\indent iv. Since $f$ is invertible we have $y = f(x) \Longleftrightarrow x = f^{-1}(y)$ so that \\ 

$\lim\limits_{x \to x_{0}} f(x) = y_{0} \Longleftrightarrow \lim\limits_{y \to y_{0}} f^{-1}(y) = x_{0} \Longleftrightarrow \lim\limits_{x \to y_{0}} f^{-1}(x) = x_{0}$   $\blacksquare$\\ 

\begin{thm}
 If the functions $f, g$ have limits at a point $x_{0}$, then \\
\begin{hEnumerateRoman}
\item $\lim\limits_{x \to x_{0}} [ f(x) + g(x) ] = \lim\limits_{x \to x_{0}} f(x) +  \lim\limits_{x \to x_{0}} g(x) $ \\

\item $\lim\limits_{x \to x_{0}} [ f(x) - g(x) ] = \lim\limits_{x \to x_{0}} f(x) -  \lim\limits_{x \to x_{0}} g(x) $ \\

\item $\lim\limits_{x \to x_{0}} [ f(x) . g(x) ] = \lim\limits_{x \to x_{0}} f(x) .  \lim\limits_{x \to x_{0}} g(x) $ \\

\item $\lim\limits_{x \to x_{0}} [ f(x) : g(x) ] = \lim\limits_{x \to x_{0}} f(x) :  \lim\limits_{x \to x_{0}} g(x) $ ( if $\lim\limits_{x \to x_{0}} g(x) \neq 0 $)\\
\end{hEnumerateRoman}
\end{thm}

\begin{proof}	Let  
\[ \lim_{x \to x_{0}} f(x)=\alpha,\qquad   \lim_{x \to x_{0}} g(x)=\beta\] 
Then given $\varepsilon >0$, there exist deleted neighbourhoods $N_{1}, N_{2}$ of $x_{0}$ such that
 \[ x\epsilon N_{1} \Rightarrow |f(x)-\alpha|<\varepsilon, \qquad x\epsilon N_{2} \Rightarrow |g(x)-\beta|<\varepsilon \] 
 Taking $N=N_{1}\cap N_{2}$, we have 
 \[ x\epsilon N \Rightarrow |f(x)-\alpha|<\varepsilon, \quad  |g(x)-\beta|<\varepsilon  \] \\ 
\begin{enumerate}[label=\alph*) ,leftmargin=*,topsep=-10pt,partopsep=0pt,parsep=0pt,itemsep=0pt]
\item \begin{eqnarray*} x\epsilon N &\Rightarrow& |f(x)+g(x)-(\alpha+\beta)| = |f(x)-\alpha+g(x)-\beta)|\\  &\leq & |f(x)-\alpha|+|g(x)-\beta)|< \varepsilon+\varepsilon=2\varepsilon \end{eqnarray*} 
Since $\varepsilon(>0)$ is arbitrary, then $f(x)+g(x)\rightarrow \alpha+\beta$ as $x\rightarrow x_{0}$. \\
\item Similarly proved.\\
\item \begin{eqnarray*}
 x\epsilon N &\Rightarrow& |f(x)g(x)-\alpha\beta|  \\
   &=& |f(x)g(x)-\alpha g(x)+\alpha g(x)-\alpha\beta| \\
   &=& |(f(x)-\alpha)g(x)+\alpha(g(x)-\beta)| \\
   &\leq& |(f(x)-\alpha)||g(x)|+|\alpha||(g(x)-\beta)|\\
   &<& \varepsilon|g(x)|+|\alpha|\varepsilon\\
   &<& \varepsilon(|\beta|+\varepsilon)+|\alpha|\varepsilon\\
 x\epsilon N&\Rightarrow&|f(x)g(x)-\alpha\beta|<(|\alpha|+|\beta|+\varepsilon)\varepsilon\rightarrow 0.\\
\end{eqnarray*}
\item \begin{equation} \lim_{x \to x_{0}} \frac{f(x)}{g(x)}=\lim_{x \to x_{0}}[f(x).\frac{1}{g(x)}]= \lim_{x \to x_{0}}f(x).\lim_{x \to x_{0}} \frac{1}{g(x)} \end{equation}

\end{enumerate}

\end{proof}
%: b1p1/080 ++++++++++++++++++++++++++++++++++++++
\hPage{b1p1/080}
% ++++++++++++++++++++++++++++++++++++++
\begin{gather*}
= \alpha \cdot \frac{1}{\beta} = \frac{\alpha}{\beta} \quad \text {(Theorem 1 c) }
\end{gather*}


\begin{cor}
Let a composite function $g \circ f$ be given. Then

$\lim\limits_{x\rightarrow x_{0}} f(x)=\alpha $ and $\lim\limits_{x\rightarrow \alpha} g(x) $ exists  $ \Rightarrow   \lim\limits_{x\rightarrow x_{0}} (g \circ f)(x)=g(\alpha) $.

\end{cor}


\begin{thm}\quad

\begin{enumerate}[label=\arabic*),leftmargin=*,topsep=-10pt,partopsep=0pt,parsep=0pt,itemsep=0pt] 
\item If $f(x)<g(x)$ holds for all x in a deleted neighbourhood $N(x_{0})$ and if f, g have limits $\alpha$, $\beta$ at $x_{0}$, then $\alpha \leq \beta$.
\item If $f(x)<u(x)<g(x)$ holds for all x $\in N(x_{0})$ and if f, g have the same limit $\ell$ at $x_{0}$, then 
\[ \lim_{x\rightarrow x_{0}} u(x) = \ell . \]
\end{enumerate} 
\end{thm}


\begin{proof}\quad

\begin{enumerate}[label=\arabic*),leftmargin=*,topsep=-10pt,partopsep=0pt,parsep=0pt,itemsep=0pt]
\item 
\begin{align*}
g(x)-f(x)>0 & \Rightarrow  \lim\limits_{x\rightarrow x_{0}} [g(x)-f(x)]\geq 0 
\Rightarrow \lim\limits_{x\rightarrow x_{0}} g(x) - \lim\limits_{x\rightarrow x_{0}} f(x)\geq 0 \\
&\Rightarrow \beta - \alpha \geq 0 
\Rightarrow \alpha \leq \beta 
\end{align*}

\item Since f, g have limits $\ell$ at $x_{0} \in D_{f} \cap D_{g}$ then there exist $N_{1}(x_{0})$, $N_{2}(x_{0})$ such that
\[ x \in N_{1}(x_{0}) \Rightarrow |f(x) -\ell| < \mathcal{E} ,\quad x \in N_{2}(x_{0}) \Rightarrow |g(x) -\ell| < \mathcal{E} \]
implying $\ell-\mathcal{E} < f(x)<\ell+\mathcal{E} $ and $\ell-\mathcal{E} < g(x)<\ell+\mathcal{E} $. Since  $f(x)<u(x)<g(x)$ we have $\ell-\mathcal{E} < u(x)<\ell+\mathcal{E} $ which implies $|u(x) -\ell| < \mathcal{E}$ or that $\lim\limits_{x\rightarrow x_{0}} u(x)=\ell$.

\end{enumerate} 
\end{proof}




% -------------------------------------------------------------------
\section*{\underline{Corollary 1}}

\begin{equation}
P(x)=\sum_{k=0}^{n} a_k x^k\Rightarrow\lim_{x \rightarrow x_0}P(x)=P(x_0)
\end{equation}
\begin{proof} 
\begin{align*}
\lim_{x \rightarrow x_0} P(x)&=\lim_{x \rightarrow x_0}  \sum_{k=0}^{n} a_k x^k\\
&= \sum_{k=0}^{n}\lim_{x \rightarrow x_0} (a_k x^k)&&\text...{(Theorem\ 2a)} \\
&= \sum_{k=0}^{n}a_k\lim_{x \rightarrow x_0}x^k&&\text...{(Theorem\ 1b)} \\
&= \sum_{}^{}a_k(\lim_{x \rightarrow x_0}x)^k&&\text...{(Theorem\ 2c)}\\
&= \sum_{}^{}a_k x_{0}^k&&\text({x \rightarrow x_0})\\
&= P(x_0) \\
\end{align*}
\end{proof}




% -------------------------------------------------------------------
\section*{\underline{Corollary 2}}

\begin{equation*}
\text{If}\ P(x)/Q(x)\text{ \ is \ a \ rational \ function \ with \ }Q(x_0)\neq 0,\ \text{then;} \\
\end{equation*}
\begin{equation}
\lim_{x \rightarrow x_0}\frac{P(x)}{Q(x)}=\frac{P(x_0)}{Q(x_0)}\\
\end{equation}
\begin{proof} 
\begin{align*}
\lim_{x \rightarrow x_0}\frac{P(x)}{Q(x)}&=\frac{\lim_{x \rightarrow x_0}P(x)}{\lim_{x \rightarrow x_0}Q(x)}\quad &&\text{(Theorem\ 2d)}\\
&=\frac{P(x_0)}{Q(x_0)}\quad &&\text{(Coroll.1)}
\end{align*}
\end{proof}
%: b1p1/082 ++++++++++++++++++++++++++++++++++++++
\hPage{b1p1/082}
% ++++++++++++++++++++++++++++++++++++++
%: b1p1/083 ++++++++++++++++++++++++++++++++++++++
\hPage{b1p1/083}
% ++++++++++++++++++++++++++++++++++++++




% -------------------------------------------------------------------
\subsection{Indeterminate forms}

If $\lim f(x) = 0$, $\lim g(x) = 0$ when $x \rightarrow x_0$ or $ x \rightarrow \infty$, the use of property
\[
\lim \dfrac{f(x)}{g(x)} = \dfrac{\lim f(x)}{\lim g(x)}
\]
does not help in getting the limit of $ f(x) / g(x) $, since the form $ 0 / 0 $ is not defined
 and may be taken as equal to any number k. Indeed, the equality $0 / 0 = k$ is equivalent to $0 = 0\cdot k$ and the latter holds true for any $k \in \mathbb{R}$. For this reason $0 / 0$ is called an \hDefined{indeterminate form}. The indeterminate forms that we encounter in this chapter are
 \[
 \dfrac{0}{0},\quad \dfrac{\infty}{\infty},\quad \infty \cdot 0,\quad \infty - \infty
 \]
 
 There are also three other which arise in considering limit of a function of the form $ f(x)^{g(x)} $, and are $0^{0}$, $1^{\infty}, \infty^{0}$. These indeterminate forms will be taken up in a later chapter where, by the use logarithms, they will be reduced to above mentioned indeterminate forms.
 \subsubsection{The indeterminate form $0 / 0$:}
 A remarkable example is the following
 \[
 \lim_{\Theta\to\infty} \dfrac{\sin \Theta }{\Theta} = \left[ \dfrac{0}{0}\right] 
 \]
 which we state as a theorem:
 \newtheorem{theorem}{Theorem}
 \begin{thm}
 If $\Theta$ is measured in radian, then
 \[
 \lim_{\Theta\to\infty} \dfrac{\sin \Theta }{\Theta} = 1\quad or\quad \lim_{\Theta\to\infty} \dfrac{\Theta }{\sin \Theta} = 1
 \]
 \end{thm}
%: b1p1/084 ++++++++++++++++++++++++++++++++++++++
\hPage{b1p1/084}
% ++++++++++++++++++++++++++++++++++++++
%: b1p1/085 ++++++++++++++++++++++++++++++++++++++
\hPage{b1p1/085}
% ++++++++++++++++++++++++++++++++++++++
%: b1p1/086 ++++++++++++++++++++++++++++++++++++++
\hPage{b1p1/086}
% ++++++++++++++++++++++++++++++++++++++
%: b1p1/087 ++++++++++++++++++++++++++++++++++++++
\hPage{b1p1/087}
% ++++++++++++++++++++++++++++++++++++++

\begin{enumerate}
\setcounter{enumi}{1}
\item $
f(n) = \begin{cases} 1 &\mbox{when } x = -1$, $x_{0} = -1 \\
x & \mbox{when} x \geq 0 \end{cases} 
 $
\item  $ h(x) = \dfrac{x - 2}{x - 2}$, $ x_0 = 2 $
\item $ \dfrac{1}{x - 1} $, $ x_0 = 1 $
\end{enumerate}

\begin{tabular}{ l r }

\end{tabular}

\textbf{Solution}

\begin{enumerate}
\item $ f $ is not defined at $ x_0 = 0 $ (finite jump) 
\begin{figure}[h]
\includegraphics[keepaspectratio=true, scale=0.15]{images/p095_fig1}
\caption{       }
\label{fig:p095_fig1}
\end{figure} 
\item $ x = -1 $ is an isolated point of $ g $.
\begin{figure}[h]
\includegraphics[keepaspectratio, scale=0.15]{images/p095_fig2}
\caption{       }
\label{fig:p095_fig2}
\end{figure} 

\item $ h $ is undefined at $ x_0 = 2 $. $ h $ having limit (=1)  at  $ x_0=2 $ the discontinuity is removable.
\begin{figure}[h]
\includegraphics[keepaspectratio, scale=0.15]{images/p095_fig3}
\caption{       }
\label{fig:p095_fig3}
\end{figure} 


\item $ k $ is undefined at $ x_0 = 1 $ (infinite jump) 
\begin{figure}
\includegraphics[keepaspectratio, scale=0.15]{images/p095_fig4}
\caption{}
\label{fig:p095_fig4}
\end{figure} 
\end{enumerate}

\begin{exmp}
Test the function $  f(x) = |x|  $ for continuity at the origin. 
\end{exmp} 

\textbf{Solution} Since $ \underset{x \rightarrow 0}{limit} |x| = 0 $ and this limit is equal to $ f(0), f $ is continuous at 0.

\begin{figure}
\includegraphics[keepaspectratio, scale=0.15]{images/p095_fig5}
\caption{}
\label{fig:p095_fig5}
\end{figure} 


\begin{exmp}
Test the function $ f(x) = [ 3x + 1 ] $ at  $ x_0 = \dfrac{1}{2} $
\end{exmp}

\textbf{Solution}
$ \underset{x \rightarrow \dfrac{1}{2}}{lim}  f(x) = 2 = f(\dfrac{1}{2})  $ \\
It is continuous.
\begin{figure}[h]
\includegraphics[keepaspectratio, scale=0.15]{images/p095_fig6}
\caption{       }
\label{fig:p095_fig6}
\end{figure} 
%: b1p1/098 ++++++++++++++++++++++++++++++++++++++
\hPage{b1p1/098}
% ++++++++++++++++++++++++++++++++++++++
\begin{enumerate}
\item[a)] 
Since f is increasing in [1,4], we have 
 \\* m=f(1) = 1/4 , M = f(4) = 4/7.  $\mu$= $\frac{1}{2}$ $\in$ [$\frac{1}{4}$,$\frac{4}{7}$]
\\* Then $\frac{x}{x+3}$ = $\frac{1}{2}$ $\implies$  c = 3 $\in$ [1 , 4]

\item[b)] 
Since g is increasing in [2 , 5] , we have
\\* m=f(2) , M=f(5) = 23, $\nu$ = 14 $\in$ [2,23]. Then
\\*$ x^2$ - 2 = 14 $\implies$ x = $\pm$ 4 , and c = 4 $\in$ [2 , 5]
\item[c)]
\includegraphics[bb=0 0 415 400, scale=0.3]{images/p98_fig1.png}
\\*From the graph
\\* m = f(-2) = f(2) = 0 ,
\\* M = f(3) = 5 .
\\*$\nu$ = $\frac{9}{4}$ $\in$ [0 , 5].   Then
\\*
\\* \[
|x^2-4| \leq \frac{9}{4} \implies x^2 - 4 = \pm \frac{9}{4}  \implies x^2 = \frac{16 \pm 9}{4}\implies
\]
\[
\\* x_{1,2} = \pm \frac{5}{2} , x_{3,4} = \pm \frac{7}{2}  \implies
\]
\[ c_{1} = - 5/2 , \quad  c_{2} = -\sqrt{7}/2 , \quad   c_{3} = \sqrt{7}/2 , \quad    c_{4} = 5/2 \in [-5/2 , 3] .
\]
\item[d)] 
Since k is defined on an open interval there will be no smallest and no largest values,
\quad \[    but \quad  \frac{1}{25} < k(x) < 1 . 
\]
\\*
\[ \mu \in (1/25 , 1) then \quad 1/x^2 = 4/9 \implies x= \pm 3/2 \implies c=3/2 \in (1,5).
\] 

\end{enumerate} 

\underline{Corollery: } If f $\in c[a,b]$  and $f(a) f(b)< 0 $ , then there exists at least one $ c \in [a, b]$ such that $f(c) = 0$ , 
in other words the equation $ f(x) = 0$ has at least one root between a  and b .
\\*\includegraphics[bb=0 0 415 400, scale=0.3]{images/p98_fig2.png}


 To find an approximate root of an equation $f(x) = 0$, in the first step, one determines an interval $ [a, b]$ on which f is continuous and 
$f(a) f(b)<0$ and the

%: b1p1/102 ++++++++++++++++++++++++++++++++++++++
\hPage{b1p1/102}
% ++++++++++++++++++++++++++++++++++++++
\textbf{113.} Show that the following functions are continuous for all $x \in R$ :\\\\
\begin{tabular}{ll}
$a)  f_{(x)} = \left\{ 
\begin{array}{ll}
x^2, & x<-1 \\
1, & x=-1 \\
x+2, & x>-1
\end{array} 
\right.
$
& $b)  g_{(x)} = \sqrt{\frac{x^2 + 2x + 3}{x^2 + x + 1}}$
\end{tabular}
\\\\\\
\textbf{114.} Show that the following functions are continuous at all x in their domain of definition:\\


\begin{tabular}{ll}
$a)  f_{(x)} = |x^2-2| \frac{x}{x} $
& $b)  g_{(x)} = \sqrt{x^2 -5x +4}$\\\\

$c)  h_{(x)} = \sqrt[3]{x+5} $
& $d)  h_{(x)} = \sqrt[4]{x^2+2}$
\end{tabular}
\\\\\\
\textbf{115.} Find the points of discontinuity and identify their types of the following functions, if any:\\

\begin{tabular}{ll}
$a)  f_{(x)} = \frac{x^2+3x-10}{x-2}$
& 
$b)  g_{(x)} = \left\{ 
\begin{array}{ll}
x^2 +3, & x<-2 \\
5-x, & x>-2
\end{array} 
\right.$\\\\\\

$c)  F_{(x)} = \left\{ 
\begin{array}{ll}
x+4, & x<2\\
7, & x=2 \\
2x+2, & x>2
\end{array} 
\right.$
&
$d) G_{(x)}=[x]-x $
\end{tabular}
\\\\\\
\textbf{116.} Same question for:\\

\begin{tabular}{ll}
$a)  f_{(x)} = \left\{ 
\begin{array}{ll}
x, & x<0\\
1, & x=0 \\
\frac{1}{1-x}, & x>0
\end{array} 
\right.$
&
$b) g_{(x)}=\frac{x}{sin x}$\\\\\\

$c) F_{(x)}=x cot x$
&
$d) G_{(x)}=\frac{tan x}{arctan x}$
\end{tabular}
\\\\\\
\textbf{117.} Find the points and type if discontinuity of the following functions in the indicated intervals, if any:\\
%: b1p1/103 ++++++++++++++++++++++++++++++++++++++
\hPage{b1p1/103}
% ++++++++++++++++++++++++++++++++++++++
\begin{enumerate}
\renewcommand{\labelenumi}{\alph{enumi})}

\item f(x) = $\dfrac{x^2+3}{|x-2|-1}$ , $\> [0,2]$

\item g(x)= $\dfrac{x}{[2x]-x}$ , [0,5]

\item h(x) = $\dfrac{sinx+cosx}{sinx-cosx}$ , $[0,\pi/2]$

\item k(x) = $\dfrac{sinx}{arcsinx}$ , $[0,\pi/2]$\\

\end {enumerate}

\noindent 118. Find the points of discontinuity  and identify their types of the following domain, if any;

\begin{enumerate}
\renewcommand{\labelenumi}{\alph{enumi})}

\item F(x) = $[sinx]$

\item A function defined by

\begin{equation*}
x^3y^2-2x^2y-xy^2+8xy+5x-y+3=0
\end{equation*}

\item GoG if $G(x)= [x^2+1] $

\item $H^{-1}$  if$ H(x) = \dfrac{1}{x+1}$\\

\end {enumerate}

\noindent 119. Find the points of discontinuity of f + g, fg,
\\f/g if

\begin{equation*}
f(x) = 2x - \dfrac{1}{x^2}, g(x)=x^2+\dfrac{1}{x^2}\\
\end{equation*}

\noindent 120. Find the points of discontinuity of fog and gof 
\\and determine their types, if any, where

\begin{enumerate}
\renewcommand{\labelenumi}{\alph{enumi})}
\item $f(x) = x^2-1, g(x) = sinx$
\item $f(x) = cosx, g(x) = \dfrac{1}{x^2+1}$\\
\end{enumerate}
%: b1p1/104 ++++++++++++++++++++++++++++++++++++++
\hPage{b1p1/104}
% ++++++++++++++++++++++++++++++++++++++
interval if they are continuous; then find x for the given value of f(x): \\
% TO DO: error
%\begin{enumerate}
%\setcounter{enumi}{121}
%\ 
%\begin{enumerate}
%	\item  $y = x^3 - 8,		[1, 2],		f(x) = -2$\\
%	\item  $y = | x + 3 | + x$,				$[-2, 3],	$		$f(x) = 0$\\
%	\item	$y = |x^2 + 4x| - 5$,	[-1, 3],	f(x) = 2\\
%	\item	$y = [	\frac{x}{5} + 1]$,	$[1, 4]$,	f(x) = 1
%\end{enumerate}
%\item Same question for\\
%\begin{enumerate}
%	\item  $f(x)  = x^2 - \Vert x^3\Vert, [-1, 0],		f(x) = 1$\\
%	\item  $f(x) = \frac{x}{x+1} $,				$[3, 7],	$		$f(x) = 5/6$\\
%	\item	$f(x) = - \frac{2-x}{x^3 + 3x}$,	[1, 5],	f(x) = 0\\
%	\item	$f(x) = \sec x + \csc x, [ \pi/6,\pi/3 ],	f(x) = 0$
%
%\end{enumerate}
%\item Find approximately a root of  f(x) = 0 in [0, 1] and determine the maximum error.
%\begin{enumerate}
%	\item  $f(x)  = 2x^2 + x - 1 = 0$\\
%	\item  $f(x) = 3x^2 + 2x - 1 = 0$\\
%
%\end{enumerate}
%\item Find the root of $x^3 - 5x - 2 = 0$ in (0, 1) approximately with an error less than 10$\%$. (Use both ways)\\
%\item Find approximate root of $\sin x + \cos x$ in $(0, \pi)$ and give the maximum error.
%\end{enumerate}

\noindent 121. Find m, M of the following functions in the given
%: b1p1/109 ++++++++++++++++++++++++++++++++++++++
\hPage{b1p1/109}
% ++++++++++++++++++++++++++++++++++++++
140.Find the interval defined by \\

\begin{hEnumerateAlpha}
\item $|x-3|\leqslant 2$
\item $|x+2| $<$ 3$
\item $|x+7| $<$ 9$
\item $|x+9| \leqslant 7$
\end{hEnumerateAlpha}


141.Express the given interval as an inequality involving an absolute value:\\

\begin{hEnumerateAlpha}
\item $(8,8)$
\item $[5,-7]$
\item $[-4,7]$
\item $(-2,5)$
\end{hEnumerateAlpha}


142.Find the set of solution of the following equation:\\

\begin{hEnumerateAlpha}
\item $|x^2-2x|-x-1=0$
\item $|x+3|-|2x-1|-x=0$
\end{hEnumerateAlpha}


143.Same question for \\

\begin{hEnumerateAlpha}
\item $ 2|x+4|-|x-2|+x=0$
\item $|x-3|-|x+1|+4=0$
\end{hEnumerateAlpha}

        144.Prove by induction:\\

\begin{hEnumerateAlpha}
\item $x^n+y^n$ is divisible by $x+y$ for $n\in \mathbb{Z}_1$
\item $x^n-y^n$ is divisible by $x-y$ for $n\in \mathbb{Z}_1$
\end{hEnumerateAlpha}

145.Prove by induction:\\

$1-\frac{1}{2}+\frac{1}{3}-\frac{1}{4}+ - \ldots +\frac{1}{2n-1} -\frac{1}{2n}=\frac{1}{n+1}+\frac{1}{n+2}+ \ldots +\frac{1}{2n},n\in \mathbb{Z}_1$ 


146.Prove by induction:\\

$ \left\| \frac{\sin nx}{\sin x} \right\| \leqslant n$ for $n\in \mathbb{Z}_0$ ,$ x \not= k\pi$


147.Given the relation $x-|y|=1$ \\

\begin{hEnumerateAlpha}
\item Sketch it
\item write its inverse
\end{hEnumerateAlpha}


148.Same question for $|x|+|y|=1$


%: b1p1/110 ++++++++++++++++++++++++++++++++++++++
\hPage{b1p1/110}
% ++++++++++++++++++++++++++++++++++++++
149. Find the inverse of the the relation \\

\hspace{1 cm}(x-y)(3y+x)+1=0. \\

150.Which ones of the following relations are symmetric? \\

\hspace{1 cm} a) \{(x,y) : $x^{2}$ + $y^{2}$ $>$ 4\} \hspace{1 cm} b) \{(x,y) : x + y $<$ 2\}   \\

\hspace{1 cm} c)\{(x,y) : x - y $>$ 1\} \hspace{1,6 cm} d) \{(x,y) : xy + 4 = 0 \} \\

\hspace{1 cm} e) \{(x,y) : $\rvert$x - y$\rvert$ $<$ 2\} \hspace{1.2 cm} f) \{(x,y) : x$y^{2}$ - 1 = 0\} \\

151.Sketch the graph of relations: \\

\hspace{1 cm} a) $\rho$ = \{(x,y) : $\rvert$y$\rvert$ - x + 1 $>$ 0\} \\

\hspace{1 cm} b) $\rho$ = \{(x,y) : $\rvert$$\rvert$x$\rvert$ - y$\rvert$ - 3 $<$ 0\} \\

152.Same question for: \\

 \hspace{1 cm} a) \{(x,y) : $\lfloor$x -2$\rfloor$ = 3, $\lfloor$y-3$\rfloor$ = 2 \}   \\ 
 
 \hspace{1 cm} b) \{(x,y) : $\lfloor$x$\rfloor$ + $\lfloor$y$\rfloor$ = 1 \}   \\ 
 
153.Sketch: \\

\hspace{1 cm} a) \{(x,y) : $\rvert$x$\rvert$  + $\rvert$x-1$\rvert$ = 3\} \\

\hspace{1 cm} b) \{(x,y) : $\rvert$y$\rvert$ - $\rvert$y-1$\rvert$ $>$ 3\} \\

\hspace{1 cm} c) \{(x,y) : $\rvert$x$\rvert$ + $\rvert$y-1$\rvert$ $<$ 3\} \\

\hspace{1 cm} d) \{(x,y) : $\rvert$y$\rvert$ - $\rvert$x-1$\rvert$ $>$ 3\} \\

154.Sketch: \\

\hspace{1 cm} a) \{(x,y) :  $\rvert$x$\rvert$ =2\} \hspace{1 cm} b) \{(x,y) :  $\lfloor$x$\rfloor$ =2\} \\

\hspace{1 cm} c) \{(x,y) :  $\rvert$x-3$\rvert$ =2, y=1\}  d) \{(x,y) :  $\lfloor$x-3$\rfloor$ =2, y=1\} \\

155.Sketch: \\

\hspace{1 cm} a) \{(x,y) : $\rvert$2x+3$\rvert$ =5, $\lfloor$y$\rfloor$ =2\}  \\

\hspace{1 cm} b) \{(x,y) : $\lfloor$2x+3$\rfloor$ =5, $\rvert$y$\rvert$ =2\}  \\

\begin{description}
\item[156.] Sketch the graphs of the relations;
\end{description}

\textbf{a)} \{$( x,y ):  \lfloor x \rfloor  \lfloor y \rfloor  = 1 \}$

\textbf{b)} \{$( x,y ):  \lfloor x \rfloor  \lfloor y \rfloor  = -1 \}$

\textbf{c)} \{$( x,y ):  \lfloor x \rfloor  \lfloor y \rfloor  = 0 \}$

\textbf{d)} \{$( x,y ):  \lfloor x \rfloor  \lfloor y \rfloor  = 4 \}$


%-----------------

\begin{description}
\item[157.] Prove
\end{description}


\textbf{a)}  $ \lfloor x \rfloor \leq  x  < \lfloor x \rfloor + 1 $ 
\hspace*{1in} \textbf{b)} $ (a) \Leftrightarrow 0 \le x - \lfloor x \rfloor < 1 $

\textbf{c)} $ \lfloor x \rfloor + \lfloor x \rfloor \le 0 $
\hspace{92pt} \textbf{d)} $ 0 \le \lfloor x \rfloor-2 \lfloor x/2 \rfloor \le 1 $

\begin{description}
\item[158.] Given the Figure a window with constant area S. The glass in rectangular 
form permits the light half of that of semicircular form . Find the amountof light 
$ l(x)$ passing through the window.(Glass in rectangular form permits amount of light $l_a$ per unit area)
\end{description}

\begin{picture}(10,5)(10,0)
\hspace{224pt} \circle{21}2x
\end{picture}

\vspace{18pt}

\begin{picture}(10,10)(10,0)
\hspace{3in}\framebox(20,30)[17]{} y
\end{picture}


\begin{description}
\item[159.] Find the area $A$ of an isocales triangle with 
equal sides $a$ and angle between them is $x$; then discuss
the contiunity of $A$ as a function of $x$. Find m and M.
\end{description}

\begin{description}
\item[160.] Find the distance function $d(m)$ of the foot of
the perpendicular from $(4,0)$ to the line $y=mx$.
Find the domain $D$ and range of this fuctions.
\end{description}

\begin{description}
\item[161.] A variable point $P$ on $(x-2)^{2} + y^{2} = 4 $ is given.
Find the sum of the distance of $P$ form the lines $ y=x$ and $ y=-x$ as a function of $x.$ 
\end{description}

\begin{description}
\item[162.] If $f(\sqrt{2}x + 3) = x^{2} + x$ , find $f(x)$.  
\end{description}

\begin{description}
\item[163.] If $f(x)=\sqrt{x^{2} +1}$, $g(x) = x/(x^{2} +1)$ , find 
\end{description}

\textbf{a)} $(fog)(x) $
\hspace{1in}\textbf{b)} $(gof)(x)$

\textbf{c)} $f^{-1}(x)$ 
\hspace{80pt}\textbf{d)} $g^{-1})(x)$

%: b1p1/112 ++++++++++++++++++++++++++++++++++++++
\hPage{b1p1/112}
% ++++++++++++++++++++++++++++++++++++++
\begin{enumerate}
\setcounter {enumi} {163}
\item If 1/p + 1/q = 1 show that f(x) = $x^{n-1}$ and g(x) = $x^{q-1}$ are inverse functions.
\item Using the data given in the Figure, compute the time t(x) for a man walking from A to B via C 
if the speed from A to C is 2km/hr and C to B is 3km/hr.
\begin{figure} [h]
\begin{center}
\includegraphics[bb=0 0 415 400, scale=0.85]{images/p112_fig1.png}
\label{fig:p112_fig1}
\caption{}

\end{center}
\end{figure}
\item Let $e_{1}(x)$, $e_{2}(x)$ be two even and $o_{1}(x)$, $o_{2}(x)$ be two odd functions.
What can be said about  evenness or oddness of
\renewcommand{\theenumi}{\Alph{enumi}}
\begin{enumerate}
\item $e_{2}$ o $e_{1}$  \item $e_{1}$ o $o_{1}$ 
\item $o_{1}$ o $e_{1}$ \item $o_{2}$ o $o_{1}$
\end{enumerate}

\item If F, G, H are three given invertible functions and f, g, h are unknown functions defined by $f$ o $F$ = $G$,
$F$ o $g$ = $G$ and $F$ o $h$ o $G$ = $H$ show that
\begin{enumerate}
\renewcommand{\theenumi}{\Alph{enumi}}
\item $f$ =  $G$ o $F^{-1}$ \item $g = F^{-1}$ o $G$
\item $h = F^{-1}$ o $H$ o $G^{-1}$
\end{enumerate}
\item Given f(x) = $\sqrt{x+1}$, g(x) = $tan^{2}x$ and h(x) = $4x^{2}$find the following:
\begin{enumerate}
\renewcommand{\theenumi}{\Alph{enumi}}
\item ($f$ o $g$ o $h$)$(\frac{\sqrt{\varPi}}{4})$ \item ($f$ o $h$ o $g$)$(\frac{\varPi}{3})$
\item ($g$ o $h$ o $f$)(3) \item ($h$ o $f$ o $g$)$(\frac{\varPi}{6})$
\end{enumerate}
\item Prove:
\begin{gather*}
csc  \frac{\varPi} {7} - csc \frac{2\varPi}{7} - csc \frac{3\varPi}{7} = 0
\end{gather*}
\item Prove:
\begin{gather*}
arctan \frac{1}{2} + arctan \frac{1}{5} + arctan \frac{1}{8} = \frac{\varPi}{4}
\end{gather*}
\item Evaluate the following
\end{enumerate}
%: b1p1/117 ++++++++++++++++++++++++++++++++++++++
\hPage{b1p1/117}
% ++++++++++++++++++++++++++++++++++++++
\begin{enumerate}
\setcounter{enumi}{157}
\item $l(x) = \frac{1}{2}(S + \frac{\pi}{2}x^2)l_0$
\addtocounter{enumi}{1}
\item $d(m) = 4m/\sqrt{1 + m^2};  D_d = \mathbb{R},  R_d = [0, 4]$.
\addtocounter{enumi}{1}
\item $f(x) = (x^4 - 12x^3 + 56x^2 - 120x + 99)/4$.
\addtocounter{enumi}{1}
\item 
\begin{enumerate}
	\item $\sqrt{2}$
	\item $\sqrt{37}$
	\item $tan^2 16$
	\item $16/3$
\end{enumerate}
\addtocounter{enumi}{1}
\item
\begin{enumerate}
	\item $\pi/2$
	\item $2\pi$
	\item $3$
	\item $\pi/3$
\end{enumerate}
Hint: Transform first the given expression into linear form such as $3tan2x - sin5x$, and then find the period.
\addtocounter{enumi}{1}
\item
\begin{enumerate}
	\item $0$
	\item $0$
\end{enumerate}
\addtocounter{enumi}{1}
\item $5$
\addtocounter{enumi}{1}
\item
\begin{enumerate}
	\item $7$
	\item $0$
	\item No limit
	\item No limit
\end{enumerate}
\addtocounter{enumi}{1}
\item
\begin{enumerate}
	\item $\mathbb{R}$
	\item $\mathbb{R}$ - $\{x: x = k/2$, $k\in\mathbb{Z}\}$
	\item Yes
	\item $x = [2y - 1]$
\end{enumerate}
\addtocounter{enumi}{1}

\item 
$x = -R \cos 2 \alpha$, 
$y = R \sin 2 \alpha$; 
$\alpha = 3 \pi / 8$
\addtocounter{enumi}{1}

\item 
$A(\alpha) = \frac{1}{2} (- \sin \alpha) \cos \alpha$; 
$m = 0$, 
$M = 1/2$
\end{enumerate}
%: b1p1/232 ++++++++++++++++++++++++++++++++++++++
\hPage{b1p1/232}
% ++++++++++++++++++++++++++++++++++++++
\begin{hEnumerateArabic}
	
	\setcounter{enumi}{207}
	\item  
	\begin{hEnumerateAlpha}
		\begin{multicols}{2}
			\item 1,
		\columnbreak
			\item -1
		\end{multicols}
	\end{hEnumerateAlpha}
	
	\setcounter{enumi}{209}
	\item $ r = (r_1 + \ldots + r_n ) / n. $
\end{hEnumerateArabic}






% =======================================================
\end{document}  

%==== templates ====

%==== environments ====

%\begin{figure}[htb]
%	\centering
%	\includegraphics[width=0.9\textwidth]{images/SD-1-1p15A}
%	\caption{Classification of complex numbers}
%	\label{fig:classificationOfComplexNumbersA}
%\end{figure}

%\begin{center}
%\begin{tabular}{cc}
%\end{tabular}
%\end{center}

%\begin{exmp}
%\begin{hSolution}
%\end{hSolution}
%\end{exmp}

%\begin{hEnumerateAlpha}
%\end{hEnumerateAlpha}

%\begin{hEnumerateRoman}
%\end{hEnumerateRoman}

%$
%\begin{bmatrix}
%\end{bmatrix}
%$

%\frac{aaaa}{bbb}
%\frac{a_{n}}{b_{n}}
%\left( aaaa \right)
%\Longrightarrow

% ++++++++++++++++++++++++++++++++++++++
\hPage{b1p1/xxx}
% ++++++++++++++++++++++++++++++++++++++

%\begin{multicols}{2}
%	bb
%\columnbreak
%	aa
%\end{multicols}
