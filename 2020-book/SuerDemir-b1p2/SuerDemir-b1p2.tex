\documentclass[11pt]{amsbook}

\usepackage[pdftex]{graphicx}
\usepackage{../HBSuerDemir}	% ------------------------

%\usepackage[update,prepend]{epstopdf}

\usepackage{epstopdf}
\epstopdfsetup{outdir=./images}
\epstopdfsetup{update} % only regenerate pdf files when eps file is newer

\setcounter{tocdepth}{3}

\usepackage{fancyhdr} % Header/Footer
\pagestyle{fancy}
%\fancyhead{}
%\fancyhead[CO,CE]{óDraftó}
\fancyfoot{}
%\fancyfoot[CO,CE]{Draft}
%\fancyfoot[RO,LE] {\thepage}
\fancyfoot[L]{\footnotesize 
Freshman Calculus by Suer \& Demir  \textbf{DRAFT} \\
\LaTeX  by Haluk Bingol 
\href{http://www.cmpe.boun.edu.tr/~bingol}
{http://www.cmpe.boun.edu.tr/~bingol} 
%\large 
%\footnotesize 
\today}
\fancyfoot[R]{{\thepage} of \pageref{LastPage}}


\begin{document}



% -------------------------------------------------------------------
\chapter{XXX}
\label{chap:FuncLimCont}




% -------------------------------------------------------------------
\section{XXX}
\label{sec:Numbers}

%\hPage{b1p2/320}
%:b1p2/320 ++++++++++++++++++++++++++++++++++++++
\hPage{b1p2/320}
% ++++++++++++++++++++++++++++++++++++++
circle (or a center) and gradually receding from or approaching \\
it.\\ Two examples of spiral are \\ 
%
1) $r=a^\theta$ (ARCHİMEDES' Spiral)\\
%
2) $a^\theta =\ln r$ or  $r=e^{a \theta}$  (logarithmic  Spiral)(See\quad Chapter \quad6.)

%
 1. An ARCHIMEDES' Spiral is the trajectory of a point $P$ 
 moving uniformly on a line 
which in turn rotating uniformly about about a point. 

Taking the center 0 of rotation 
as pole and that line when $P$ is at 0
as polar axis, we have
\begin{figure}[h]
	\includegraphics{images/B1P2_320_F1.jpg}
\end{figure}
\begin{align*}
	r &= bt \quad \text{(linear motion on 0r)}\\
	\theta &= wt \quad \text{(uniform rotation)}\\
	r &= bt 
		= b \frac{\theta}{\omega}
		= \frac{b}{\omega}\theta
		= a\theta
\end{align*}
If $a > 0$, 
having $r \rightarrow \infty $ as $ \theta \rightarrow \infty $, 
the curve is a spiral admitting CPA as axis of symmetry 
(since $r \rightarrow -r$ when $\theta \rightarrow - \theta$ ), 
with PA as tangent at 0. (when $a<0$, the motion takes place
in clockwise sense ).

In the history of mathematics this curve is the first curve other than
the circle to which tangent line has been constructed. 
Also Archimedes used this curve to trisect an angle and squaring a circle.

%
2. The curve $r=e^{a\theta} \quad (a>0)$ is another spiral 
since $r \rightarrow $ as $\theta \rightarrow \infty$. 
Furthermore since 
$r \rightarrow 0$ as $\theta \rightarrow -\infty $, 
the pole is a point-asymptote.
\begin{figure}[h]
	\includegraphics{images/B1P2_320_F2.png}\\
\end{figure}
Among other spirals we mention the following.

%
3. $r \theta = a$ (hyperbolic or reciprocal spiral)



%:b1p2/331 ++++++++++++++++++++++++++++++++++++++
\hPage{b1p2/331}
% ++++++++++++++++++++++++++++++++++++++
\[
	\frac{a}{b} = c \text{ or }  a = bc
\]
in view of \refeq{eq:b1p330eq1}, one gets
\begin{align}
	\label{eq:b1p331eq2}
	%
	\hAbs{a} = \hAbs{b} \hAbs{c}, 
	\quad 
	\arg a = \arg b + \arg c\\
	\text{or}\\ % \nolabel\\
	\hAbs{\frac{a}{b}} = \frac{{}\hAbs{a}}{\hAbs{b}} 
	\quad \text{and} 
	\quad
	 \arg \frac{a}{b} = \arg a - \arg  b
\end{align}
In words, 
\textit{
	the modulus of the ratio of two complex numbers is 
	the ratio of their moduli, and 
	argument is the difference of their arguments.
}

%
\begin{exmp}
	Given the complex numbers	
	\[
		u = 6 + 2 \hI,  v = 4 + 2 \hI
	\]
	find the polar form of their product.
	
	\begin{enumerate}[label=(\alph*)]
		
		\item 
		by the property \refeq{b1p330eq1}
		
		\item 
		first finding the cartesian product, and 
		then transforming to polar form.
	\end{enumerate}
	%
	\begin{hSolution}
		\begin{enumerate}[label=(\alph*)]
			
			\item 
			\begin{align*}
				\hAbs{uv} 
					= \hAbs{u}\hAbs{v},
				\quad 
				\hAbs{uv} 
					= \sqrt{40} \sqrt{20} 
					= 20\sqrt{2},\\
				\arg (uv) = \arg u + \arg v 
					= \arctan \frac{1}{3} + \arctan \frac{1}{2}.\\
				 tan \left( \arctan \frac{1}{3} + \arctan \frac{1}{2} \right) 
					 = \frac{1/3 + 1/2}{1 - 1/6} 
					 = 1\\
				 \Rightarrow \arg (uv) = \frac{\pi}{4} +  k \pi\\
				 \Rightarrow \hArg (uv) = \pi / 4, \text{ since } \Im (uv) > 0.\\
				 \Rightarrow uv = 20 \sqrt{2} 
				 	\left( \cos \frac{\pi}{4} + \hI \sin \frac{\pi}{4} \right)
			\end{align*}
			
			\item 
			\begin{align*}
				uv = 20 + 20 \hI 
					\Rightarrow  \hAbs{uv} = 20\sqrt{2},\\
				\arg (uv) = \arctan 1\\
				\Rightarrow 
					\arg (uv) 
					= \frac{\pi}{4} + k\pi 
					= \frac{\pi}{4}, 
				\text{ since } 
				\Re (uv) > 0, 
				\Im (uv) > 0.\\
				\Rightarrow 
					uv = 
					20 \sqrt{2} \left( \cos \frac{\pi}{4} 
					+ \hI \sin \frac{\pi}{4} \right)
			\end{align*}
		\end{enumerate}
	\end{hSolution}
\end{exmp}
%
\begin{exmp}
	Given the complex numbers	
\end{exmp}



%: b1p2/333 ++++++++++++++++++++++++++++++++++++++
\hPage{b1p2/333}
% ++++++++++++++++++++++++++++++++++++++
% =======================================
\[
    \xi = \rho(\cos\psi+ \hI \sin\psi)
\]
is an $n$th root of the complex number
\[
    z = r (\cos\theta + \hI \sin\theta),
\]
it satisfies the equation $\xi^n = z$ which, 
by the use of De Moivre's formula, 
gives 
\[
    \rho^n (\cos  n \psi + \hI \sin n \psi 
    = r \cos ( \theta + 2 k \pi) + \hI \sin ( \theta + 2 k \pi)   
\]
and there
\[
	\rho^n = r, 
	\quad  
	n \psi = \theta + 2 k \pi
\]
\[
	\rho = \sqrt [n] {r},
	\enspace 
	\psi = \frac{\theta}{n} + k \frac{2 \pi}{n} ,
	\enspace 
	k \in  \mathbb{Z}.
\]
Hence $z$ has $n$ distinct roots given by
\[
	z_{k} = \sqrt[n]{r} 
	\left( 
		\cos \left( \frac{\theta}{n} + k \frac{2 \pi}{n} \right)
		+ \hI \sin \left( \frac{\theta}{n} +k \frac{2 \pi}{n} \right) 
	\right) , 
	\quad
	k = 1, 2, \dotsc, n.
\]

Since $\hAbs{z_{k}} = \sqrt[n]{r}$, 
all these roots $z_{1}, z_{2}, \dotsc, z_{n}$ lie on the circle 
with center at the origin and radius  $\sqrt[n]{r}$  
as vertices of a regular $n$-gon.

In particular, the $n$th roots of $1$ (unity) are 
\[
    \varepsilon_{k} = \cos k \frac{2 \pi}{n}
    	+ \hI \sin k \frac{2 \pi}{n},
	\quad
	k = 1, 2, \dotsc, n.
\]
one of which, 
namely $\varepsilon_{n}$,  
is the number $1$ 
(Note that  $\varepsilon_{n} = \varepsilon_{0}$).

If one of the $n$th roots is of $z$ is $z_{1}$, 
then all the $n$th roots of $z$ are obtained multiplying $z_{1}$ by $\varepsilon_{1}, \dotsc, \varepsilon_{n}$.

The roots of the polynomial equation $z^n - 1 = 0$ being
$\varepsilon_{1} , \dots , \varepsilon_{n}$, 
the following properties are the consequences of the relations 
between the roots and coefficients:
\begin{align*}
	\sigma_{1} 
		&= \sum \varepsilon_{k}
		= \varepsilon_{1} + \dotsb +\varepsilon_{n}
		= 0,\\
	\sigma_{2} 
		&= \sum_{k < l} \varepsilon_{k} \varepsilon_{l}
		= \varepsilon_{1} \varepsilon_{2} 
			+ \dotsb 
			+ \varepsilon_{1} \varepsilon_{n} 
			+ \dotsb
			+ \varepsilon_{n-1} \varepsilon_{n}
		= 0,\\
    \sigma_{3} 
		&= \sum_{k < l <m} \varepsilon_{k}\varepsilon_{l}\varepsilon_{m} 
		= 0,\\
		&\vdots\\
	\sigma_{n-1}
		&= 0.
\end{align*}
%: b1p2/423 ++++++++++++++++++++++++++++++++++++++
\hPage{b1p2/423}
% ++++++++++++++++++++++++++++++++++++++
%\footnote{Package used: mdframed for the box without bottom part}
%\footnote{Package used: mathbx to get $ \Rightarrow$ (instead we can use $\Rightarrow$ )}
%\footnote{Package used: graphbox to align the images}
%\footnote{Set the mdframe option:newmdenv[bottomline=false]\{notbottom\}}}

\begin{hEnumerateArabic}
    \item[]
        \begin{hEnumerateAlpha}
            \item 
            $ (Ch x + Sh x )^{- Argch x} 
            	= { ( ch x - Sh x )}^{- Argsh (x-2)}  $
            \item \( Ch 7x + Ch 5x + Ch 3x = Ch 6x + Ch 4x + Ch 2x \)
        \end{hEnumerateAlpha}
\end{hEnumerateArabic}
\section*{ANSWERS TO EVEN NUMBERED EXERCISES}
\begin{hEnumerateArabic}
    \setcounter{enumi}{5}
    \item 24/7, 25/7, 24/25 , 7/25, 7/24, 25/24.
    \setcounter{enumi}{47}
    \item 
        \begin{hEnumerateAlpha}
            \item R, $(2x + 2) Ch(x^2 + 2)$,
            \item R, $-(2x - 2) Sech(x^2 - 2x) Th(x^2 - 2x)$
        \end{hEnumerateAlpha}
    \setcounter{enumi}{51}
    \item 
        \begin{hEnumerateAlpha}
            \begin{multicols}{2}
                \item $-\csc x$,
                \columnbreak
                \item $\csc x$
            \end{multicols}
        \end{hEnumerateAlpha}
    \setcounter{enumi}{53}
    \item
%        \begin{hEnumerateAlpha}
%            \begin{multicols}{2}
%                \item 
%                \includegraphics[align=t,scale=0.75]{images/b1p2-423-fig01.png}
%                \columnbreak
%                \item 
%                \includegraphics[align=t,scale=0.75]{images/b1p2-423-fig02.png}
%            \end{multicols}
%        \end{hEnumerateAlpha}
        \begin{hEnumerateAlpha}
            \begin{multicols}{2}
                \item 
                \includegraphics[scale=0.75]{images/b1p2-423-fig01.png}
                \columnbreak
                \item 
                \includegraphics[scale=0.75]{images/b1p2-423-fig02.png}
            \end{multicols}
        \end{hEnumerateAlpha}



    \setcounter{enumi}{57}
    \item
        \begin{hEnumerateAlpha}
            \begin{multicols}{2}
                \item \( \ln \sqrt{3}\)  
                \columnbreak
                \item 0
            \end{multicols}
        \end{hEnumerateAlpha}
    \setcounter{enumi}{59}
    \item
        \begin{hEnumerateAlpha}
            \begin{multicols}{2}
                \item 5/4  
                \columnbreak
                \item 0
            \end{multicols}
        \end{hEnumerateAlpha}
\end{hEnumerateArabic}
\section*{A SUMMARY}
%\begin{notbottom}
    \begin{hEnumerateArabic}
        \item[6.1]
            \begin{hEnumerateAlpha}{}
                \item[] \(\ln x = \int_{1}^{x} \frac{dt}{t} (x > 0), \ln 1 = 0, \ln e = 1 \)
                \item[] \(\ln ab = \ln a + \ln b, \ln \frac{a}{b} = \ln a - \ln b\)
                \item[] \(\log_b x = \log_a x \cdot \log_b a\) (change of base)
                \item[] \(\frac{d}{dx} a^{u(x)} = a^u \frac{du}{dx} \ln u, \frac{d}{dx} \log_a u(x) = \frac{1}{u} \frac{du}{dx} \log e  \)
                \item[] \(\frac{d}{dx} u(x)^{v(x)} = u^v \big(\frac{dv}{dx} \ln u +  \frac{v}{u} \frac{du}{dx} \big)  \)
                \item[] \( y = uv \ldots w  \Rightarrow \frac{y^\prime}{y} = \frac{u^\prime}{u} + \frac{v^\prime}{v} + \ldots + \frac{w^\prime}{w} \text{(logarithmic derivative)} \)
            \end{hEnumerateAlpha}
    \end{hEnumerateArabic}
%\end{notbottom}




% =========================================
\end{document}  

%==== templates ====

%==== environments ====

%\begin{figure}[htb]
%	\centering
%	\includegraphics[width=0.9\textwidth]{images/SD-1-1p15A}
%	\caption{Classification of complex numbers}
%	\label{fig:classificationOfComplexNumbersA}
%\end{figure}

%\begin{center}
%\begin{tabular}{cc}
%\end{tabular}
%\end{center}

%\begin{exmp}
%\begin{hSolution}
%\end{hSolution}
%\end{exmp}

%\begin{hEnumerateAlpha}
%\end{hEnumerateAlpha}

%\begin{hEnumerateRoman}
%\end{hEnumerateRoman}

%$
%\begin{bmatrix}
%\end{bmatrix}
%$

%\frac{aaaa}{bbb}
%\frac{a_{n}}{b_{n}}
%\left( aaaa \right)
%\Longrightarrow

% ++++++++++++++++++++++++++++++++++++++
\hPage{b1p1/xxx}
% ++++++++++++++++++++++++++++++++++++++

%\begin{multicols}{2}
%	bb
%\columnbreak
%	aa
%\end{multicols}
